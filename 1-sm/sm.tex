\chapter{The Standard Model}
\label{sec:sm}
\section{Introduction}
The \acl{SM} of particle physics is the best available theory describing the
interactions of all known fundamental
particles~\cite{aitchison,griffiths,peskin_schroeder,sm_intro,glashow,weinberg,salam,englert,higgs}. It
is perhaps the most extensively tested of any fundamental theory of nature,
having withstood the scrutiny of several decades of data-taking at a multitude
of high-energy physics experiments. However, despite this success, there appear
to be a number of major theoretical deficiencies in desperate need of
attention. In this chapter, the foundations of the theory will be laid out in a
concise manner. As well as being of direct relevance to the measurement
described in \chap~\ref{sec:wpol}, this will inform discussion of the
aforementioned theoretical issues, and potential solutions, in
\chap~\ref{sec:susy}.

It is important to note that, although the \ac{SM} will be presented here as a
complete theory, seemingly designed to match the currently observed set of
fundamental particles, this is not how it came into being. The theory was built
up over much of the second half of the 20th century and actually successfully
predicted a number of discoveries. Perhaps the best example of this is the
discovery of the \PW and \PZ bosons by the UA1 and UA2 experiments at \ac{CERN}
in 1983~\cite{ua1_w, ua1_z}. These had been theorised by Weinberg,
Glashow and Salam in 1968~\cite{weinberg,glashow,salam}.

\section{Particles and Fundamental Forces}
\label{sec:theory:particles}
The \ac{SM} incorporates three of the four known fundamental forces, namely:
electromagnetism, the weak nuclear force and the strong nuclear force. The other
force, gravity, having resisted unification with the other three, is not part of
the \ac{SM}.

Within the \ac{SM}, each fundamental force is mediated by one or more ``gauge
bosons''. For electromagnetism, this is the photon, for the weak force, the
\PWp, \PWm and \PZ bosons. Strong interactions are mediated by a set of 8
``gluons'' (\Pg). The bosons are all vector particles with a spin of 1.

As a complete theory of the known fundamental particles, the \ac{SM} must also
describe the many matter particles discovered by eexperiment. These are the
``fermions'', of which there are two varieties -- leptons and quarks. The
leptons, or ``light particles'', make up three generations. Each generation
associates a relatively heavy charged lepton with a much lighter neutral
partner, a ``neutrino''. The charged leptons are: the electron, \Pe, the muon,
\Pgm, and the tau lepton, \Ptau. Each has identical charge, typically written in
terms of the charge carried by a single electron, \ec. The corresponding neutral
leptons are then referred to as the electron neutrino, \Pnue, muon neutrino,
\Pnum, and tau neutrino, \Pnut.

The quarks are also arranged in three generations, with two quarks occupying
each. The first of each generation is referred to as an ``up-type'' quark and
has charge $+\frac{2}{3}\ec$, the second ``down-type'' with charge
$-\frac{1}{3}\ec$. The up-type quarks are (in order of increasing mass): up
(\Pup), charm (\Pcharm) and top (\Ptop)\footnote{also known as
  truth}. Similarly, the respective down-type partners are: down (\Pdown),
strange (\Pstrange) and bottom\footnote{also known as beauty} (\Pbottom). In
addition to electromagnetism and the weak force, quarks interact via the strong
nuclear force.

Each of the fermions is associated with an anti-particle partner, with opposite
charge. For the charged leptons, the charge may be indicated by a superscript,
e.g. \Pep, \Pem, \Pgmp, \Pgmm etc. For the quarks, the anti-particles will
normally be denoted \APup, \APdown etc. In the case of the neutrinos, being
electromagnetically neutral, the question of the nature of the anti-particles is
as yet unanswered. It is theoretically possible for a neutrino to be its own
anti-particle, $\Pnulepton = \APnulepton$. This would make the neutrino a
``Majorana Fermion''~\cite{majorana} and is an area of active experimental
research~\cite{majorana_neutrinos}.

The particle content and properties of the \ac{SM} seem, at first glance, to be
quite arbitrary. Why are there three forces? Why does the Weak sector have three
bosons, and the electromagnetic only one. Some of these questions may be
answered by constructing the \ac{SM} as a so-called ``gauge theory''. This will
be explained in the next section.

\section{Electroweak Gauge Theory}
\subsection{Gauge Invariance}
The theoretical principle of gauge invariance can be seen in Maxwell's theory of
electromagnetism~\cite{aitchison}. Recall that the electric (\emE) and magnetic
(\emB) components of the field may be written in terms of a vector potential
(\emA) and a scalar potential \emV as follows:
\begin{eqnarray*}
\emB = \nablav \times \emA\quad\textrm{and} \\
\emE = -\nablav \emV - \frac{\partial \emA}{\partial t}.
\end{eqnarray*}
It can be seen that these equations do not relate a given (\emB, \emE) to unique
values of \emA and \emV. In particular, if the following transformations are
applied simultaneously to \emA and \emV, the values of \emB and \emE will be
unchanged,
\begin{eqnarray*}
\emA \longrightarrow \emA + \nablav \chi\quad\textrm{and}\\
\emV \longrightarrow  \emV - \frac{\partial \chi}{\partial t},
\end{eqnarray*}
where $\chi$ is some arbitrary function. Such transformations are known as
``gauge transformations'' and \emB and \emE are said to be ``gauge
invariant''.

If the scalar and vector potentials are rewritten in terms of a single 4-vector
potential, $\Amu = (\emV, \emA)$ and taking a 4-vector differential operator,
$\dmu = (\partial/\partial t, -\nablav)$, the gauge transformation takes the
form
\begin{equation}
\label{eqn:theory_gauge_transform}
\Amu \longrightarrow \Amu - \dmu\chi.
\end{equation}
In this form, the Maxwell equations can be rewritten as
\begin{equation*}
\dmu \Fmunu = j^{\nu}_{\textrm{em}},
\end{equation*}
where $\Fmunu$ is the electromagnetic tensor, defined as
\begin{equation*}
\Fmunu = \dmu\Anu - \dnu\Amu.
\end{equation*}
and $j^{\nu}_{\textrm{em}}$ is the ``current''. It can be seen that $\Fmunu$ is
invariant under the gauge transformation (\eqn~\ref{eqn:theory_gauge_transform})
and thus the Maxwell equations are gauge invariant.

\subsection{The Principle of Least Action and Lagrangian Formalism}
Before continuing, it is useful to review the Lagrangian formalism and the
principle of least action. The action, \action, is a quantity associated with a
physical system, used to characterise its dynamics. The action can be written
\begin{equation*}
  S = \int dt \lagrangian = \int d^4 x \lagrangiand,
\end{equation*}
where \lagrangian is the \emph{Lagrangian}, and \lagrangiand the
\emph{Lagrangian density} and
\begin{equation*}
\lagrangian = \int d^3 x \lagrangiand.
\end{equation*}
The Lagrangian is a function of some generalised coordinates \phii describing
the dynamics of the system and their derivatives $\dmu \phii$. From this, the
Euler-Lagrange relation can be used to derive an equation of motion for the
system~\cite{peskin_schroeder},
\begin{equation*}
\frac{\partial \lagrangiand}{\partial \phii} - \partial_{\mu} \left (
  \frac{\partial \lagrangiand}{\partial\left(\partial_{\mu} \phii\right)}\right) = 0.
\end{equation*}
By writing down a suitable Lagrangian, a theory may be fully described. However,
inn order to perform calculation of real, measurable quantities, the theory must
first be quantised. This is a complex procedure involving \emph{renormalisation}
techniques which are beyond the scope of this discussion. We shall only note
that renormalisability requires a theory with dimensionality at most $[M]^4$
i.e. mass to the fourth power~\cite{peskin_schroeder}.

\subsection{A Real Scalar Field}
To see how a theory can be written in terms of a Lagrangian, we will first
approach a highly simplified example, namely that of a \emph{real scalar
  field}. The Lagrangian density for such a theory may be written as follows:
\begin{equation}
\label{eqn:theory_phi4}
\lagrangiand =
\frac{1}{2}\left(\partial_{\mu}\phi(x))(\partial^{\mu}\phi(x)\right) -
\frac{1}{2}m^2\phi^2(x) - \frac{\lambda}{4!}\phi^4(x).
\end{equation}
The three components of this expression are known respectively as the
\emph{kinetic term}, the \emph{mass term} and the \emph{interaction term}. The
field $\phi$ is a function of spacetime coordinates $x$ representing a single
kind of particle $\phi$, $m$ is the mass of the particle and $\lambda$ a
constant controlling the strength of the interaction between particles in the
theory.

Setting $\lambda = 0$ and applying the Euler-Lagrange relations to
Equation~\ref{eqn:theory_phi4}, one obtains
\begin{equation*}
\partial^{\mu}\partial_{\mu} \phi + m^2\phi = 0.
\end{equation*}
This can be recognised as the Klein-Gordon equation describing the motion of a
scalar particle.

\subsection{Symmetries}
It can be seen that the Lagrangian of \eqn~\ref{eqn:theory_phi4} is
invariant under certain transformations. In particular, it has been constructed
to obey the symmetry $\phi \longrightarrow -\phi$. Such symmetries are known as
global symmetries of the theory -- global in the sense that they are performed
identically at all points in spacetime. These symmetries can be classified in
terms of the mathematical theory of groups.

Noether's theorem states, that for a theory with a continuous symmetry, each
generator of the symmetry group corresponds to a conserved current in the
theory~\cite{qft_nutshell}. This elucidates many aspects of the physics of a
given theory by considering only its symmetries.

\subsection{Complex Scalar Fields and the Gauge Principle}
We will now take a slightly more realistic example -- that of a complex scalar
field. The Lagrangian for this theory may be written as follows,
\begin{equation}
\label{eqn:theory_complex_scalar}
\lagrangiand =
\frac{1}{2}\left(\partial_{\mu}\phi(x))(\partial^{\mu}\phi(x)\right)^* -
\frac{1}{2}m^2\left|\phi(x)\right|^2 - \frac{\lambda}{4!}\left(\left|\phi(x)\right|^2\right),
\end{equation}
where the field $\phi$ is now a complex quantity. This field is seen to possess
an additional symmetry,
\begin{equation}
\label{eqn:theory_phase_transform}
\phi \longrightarrow e^{i\theta}\phi,
\end{equation}
where $\theta$ is some arbitrary parameter (constant, for the moment, in
spacetime). This is known as a global \Uone symmetry and has a single generator
- and therefore, by Noether's theorem, a single conserved current.

The transformation of \eqn~\ref{eqn:theory_phase_transform} can be extended such
that the parameter $\theta$ becomes a function of the spacetime coordinate, $x$
- a ``local'' symmetry transformation. The Lagrangian is then no longer
invariant with respect to this transformation. The symmetry is said to be
``broken'',
\begin{equation}
\label{eqn:theory_gauge_dL}
\delta\lagrangiand = \left(\partial_{\mu}
  \theta\right) \left[i\left(\partial^{\mu} \phi^*\right)\phi
  -i\phi^*\left(\partial^{\mu} \phi\right)\right] +
\left(\partial_{\mu}\theta\right)\left(\partial^{\mu}\theta\right)\left|\phi\right|^2.
\end{equation}
In order to compensate for this change, we define the ``covariant derivative''
as follows
\begin{equation}
\label{eqn:theory_cov_deriv}
D_{\mu} = \partial_{\mu} + igB_{\mu}(x),
\end{equation}
where $B$ is taken to be a new four-vector field and $g$ a constant. This
operator must transform under a change of phase in such a way as to cancel the
additional terms in \eqn~\ref{eqn:theory_gauge_dL}. The Lagrangian would then be
rewritten as follows,
\begin{equation*}
  \lagrangiand = \left(D_{\mu}\phi\right)\left(D^{\mu}\phi\right)^*
  - \frac{1}{2}m^2\left|\phi(x)\right|^2 - \frac{\lambda}{4!}\left(\left|\phi(x)\right|^2\right).
\end{equation*}
It can be shown that this requires the $B_{\mu}$ field to transform as follows
\begin{equation*}
B_{\mu} \longrightarrow B'_{\mu} = B_{\mu} - \frac{1}{g}\partial_{\mu}\theta.
\end{equation*}
This is seen to be the same transformation as that applied to the
electromagnetic four-vector field in \eqn~\ref{eqn:theory_gauge_transform}. By
requiring the fields to be symmetric under a local \Uone phase transformation, a
new field must be introduced. This field then appears as a gauge boson in the
theory -- in this case an analogue of the photon. Incorporating the Maxwell
equations into the Lagrangian, one arrives at \acf{SQED},
\begin{equation}
\label{eqn:scalar_qed}
  \lagrangiand = -\frac{1}{4}F_{\mu\nu}F^{\mu\nu} + \left(D_{\mu}\phi\right)\left(D^{\mu}\phi\right)^*
  - \frac{1}{2}m^2\left|\phi(x)\right|^2 -
  \frac{\lambda}{4!}\left(\left|\phi(x)\right|^2\right).
\end{equation}

The procedure which has just been outlined is an example of the \emph{gauge
  principle}~\cite{sm_intro}. This might seem an interesting, but trivial
result, but it turns out that all particles and interactions of the \ac{SM} (see
\sec~\ref{sec:theory:particles}) can be predicted by essentially repeating this
procedure with larger symmetry groups.

\subsection{Yang-Mills Theory}
We have seen how the gauge principle can be applied, within a highly simplified
model, to predict the existence of gauge bosons by enforcing a local
symmetry. This will now be extended to ``larger'' symmetry groups.

Extending the symmetry group to \SUtwo and replacing the single complex field
$\phi$ by a vector of complex fields, $\Phi$,
\begin{eqnarray*}
\Phi \longrightarrow \Phi' = \exp\left(i\theta(x) + iT^a W^a_{\mu}\right)\Phi\\
D_{\mu} = \partial_{\mu} + igT^a_{\mu},
\end{eqnarray*}
where $T^a$ are the generators of the group \SUtwo, $T^a = \frac{1}{2}\sigma^a$
and the $\sigma^a$ are the well-known Pauli matrices. Notice that this symmetry
group predicts the existence of three gauge bosons -- matching the three weak
gauge bosons: \PZ, \PWp and \PWm. This is an example of a Yang-Mills
theory~\cite{yangmills}.

An important aspect of this toy theory is related to the structure of the
symmetry group \SUtwo. In particular, the generator of this symmetry group do
not commute -- they are said to be ``non-Abelian''. This requires the Lagrangian
to be modifed to maintain local gauge invariance,
\begin{eqnarray*}
\lagrangiand = -\frac{1}{4}F^a_{\mu\nu}F^{\mu\nu} \textrm{where}\\
F^a_{\mu\nu} = \partial_{\mu} A^a_{\nu} - \partial_{\nu} A^a_{\mu} + g f^{abc}
A^b_{\mu} A^c_{\nu},
\end{eqnarray*}
and $f^{abc}$ are known as \emph{structure constants} of the group \SUtwo. The
additional quadratic terms in the Lagrangian lead to self interactions, just as
is observed in the weak sector.

It would seem that a model based on a symmetry group $\SUtwo\times\Uone$ would
appear to give the correct number of degrees of freedom for a unification of the
electromagnetic and weak nuclear forces. One immediate problem is the fact that
the weak bosons are known to be massive. It is not possible to introduce
straightforward mass terms into the Lagrangian whilst maintaing gauge
invariance. Another more subtle issue is that the \PWp and \PWm would be
uncharged in this picture, since the \SUtwo component is able to commute with
the \Uone part. % Refs here would be good

\subsection{Spin and Chirality}
Having arrived at a toy gauge theory bearing some similarity to the \ac{SM}, it
is important to introduce a concept not yet represented in the theory. The
Lagrangians presented so far have worked only with scalar fields, but these are
known to not represent the physical world. In reality, particles have intrinsic
angular momentum or \emph{spin}.

Fermionic degrees of freedom are represented as \emph{spinors}. These transform
in a different manner to vectors under spatial rotations. In addition, they obey
a different equation of motion, known as the \emph{Dirac
  Equation}~\cite{peskin_schroeder},
\begin{equation*}
i\gamma^\mu \partial_{\mu}\Psi + m\Psi = 0,
\end{equation*}
where $\Psi$ is a spinor and $\gamma^{\mu}$ are known as the Dirac matrices (see
e.g.~\cite{aitchison}). The Dirac matrices obey the anti-commutation relation
$\{\gamma^{\mu}, \gamma^{\nu}\} = 2g^{\mu\nu}$ where $g^{\mu\nu}$ is the
spacetime metric. A suitable Lagrangian may be written as
\begin{equation*}
\bar{\Psi} \left (i\gamma^{\mu}\partial_{\mu} -m\right)\Psi \qquad \textrm{and} \qquad \bar{\Psi} =
\Psi^{\dagger}\gamma_0.
\end{equation*}
An important aspect of spinors is a property known as ``handedness'' or
\emph{chirality}. Spinors have both ``left-handed'' and ``right-handed''
components. A particle's chirality is related to its transformation properties
under the \Poincare group -- the symmetry group of spacetime.

Physically speaking, a right-handed particle is one whose spin
is aligned with its direction of motion and a left-handed one whose spin is
oppositely aligned. Note that a left-handed anti-spinor corresponds to a right
handed physical particle~\cite{peskin_schroeder}.

A related, but distinct concept, is \emph{helicity}. Helicity is the projection
of a particle's spin, $\vec{S}$ onto its momentum, $\vec{p}$~\cite{peskin_schroeder},
\begin{equation*}
h = \frac{\vec{S}.\vec{p}}{\left|\vec{p}\right|}.
\end{equation*}
In cases where the particle's spin is aligned with its momentum, it is said to
have right-handed helicity. In cases where it is anti-aligned, it has
left-handed helicity. For massless particles, the helicity and chirality states
are equivalent, and the helicity must be non-zero. Particles with helicity,
$h=\pm 1$, are said to be \emph{transversely polarised} and $h=0$,
\emph{longitudinally polarised}.
%TODO: refer to this bit

\subsection{The Electroweak Theory}\label{sec:sm_electroweak}
A striking property of the electroweak theory is the observation of parity
violation by Wu et al.~\cite{wu_parity} in the $\beta$ decay of Cobalt-60
atoms. The spins of the atoms were polarised by a magnetic field. The angular
distribution of the $\beta$ electrons was then measured. It was seen that
electrons were emitted preferentially in the direction opposite to their
spin. This implies that the neutrino momentum in the decay is always opposite to
its spin -- i.e. the neutrino is left-handed. More generally, the weak gauge
bosons interact only with left-handed particles and right-handed
anti-particles. Parity is thus said to be \emph{maximally} violated. The
coupling of the weak bosons has a ``vector minus axial'' or \VminusA form which
leads to parity violation. Parity violation will be of vital importance to the
analysis presented in \chap~\ref{sec:wpol}.

It is not necessary for us to fully detail the electroweak sector of the \ac{SM}
Lagrangian. For the sake of completeness, an outline of the construction of the
theory will be given. Firstly, the symmetry group chosen is
$\SUtwo_{\textrm{L}}\times \Uone_{Y}$ where the subscript $L$ indicates a
coupling only to left-handed spinors. The $Y$ refers to the fact that the
$\Uone$ group here is not electromagnetism but ``hypercharge''. This is
essential to overcoming the issues with $\SUtwo\times\Uone$ which have already
been described.

The fermions themselves are placed into ``isospin'' doublets, coupling a charged
lepton with a neutrino~\cite{sm_intro},
\begin{equation*}
\PsiL = \left(\begin{array}{c} \Pgn_{\textrm{L}} \\
    \Plepton_{\textrm{L}} \end{array}\right),
\end{equation*}
where $\Pgn_{\textrm{L}}$ and $\Plepton_{\textrm{L}}$ are left-handed spinor
fields representing a neutral and charged lepton respectively. The right-handed
component of the charged lepton (which \emph{does} interact electromagnetically)
is incorporated as an isospin singlet,
\begin{equation*}
\PsiR = \Plepton_{\textrm{R}}.
\end{equation*}


Dividing the Lagrangian as follows,
\begin{equation*}
\lagrangiand = \lagrangiand_{\textrm{gauge}} + \lagrangiand_{\textrm{free}},
\end{equation*}
and writing the analogue of the tensor, \Fmunu, for the gauge group
$\SUtwo_{\textrm{L}}\times\Uone$, one obtains
\begin{equation*}
\lagrangiand_{\textrm{gauge}} = -\frac{1}{4} W^{a\mu\nu} W^a_{\mu\nu}
-\frac{1}{4} B^{\mu\nu}B_{\mu\nu},
\end{equation*}
where
\begin{eqnarray*}
W^{a}_{\mu\nu} = \partial_{\mu} W^a_{\nu} - \partial_{\nu}W^a_{\mu} + g f^{abc}
W^{b}_{\mu} W^c_{\nu}\quad\textrm{and}\\
B_{\mu\nu} = \partial_{\mu} B_{\nu} - \partial_{\nu} B_{\mu}.
\end{eqnarray*}
The piece of the Lagrangian for free leptons is then,
\begin{equation*}
\lagrangiand_{\textrm{leptons}} = \APsiL i \gamma^{\mu} D_{\mu} \PsiL + \APsiR i
\gamma^{\mu} D_{\mu} \PsiR.
\end{equation*}
with the covariant derivative
\begin{equation*}
D_{\mu} = \partial_{\mu} + i g W^a_{\mu} T^a + i g' B^{\mu}\frac{Y}{2},
\end{equation*}
where $T^a = \frac{1}{2}\sigma^a$ when acting on a left-handed spinor and
zero otherwise. Similarly, the hypercharge $Y$ is $-1$ for left-handed spinors
and $-2$ for right-handed. The $g$ and $g'$ are coupling constants.

The physical \PWp and \PWm bosons are superpositions~\cite{sm_intro},
\begin{equation*}
\PWpm_{\mu} = \frac{1}{\sqrt{2}}\left(\PW^1_{\mu} + \PW^2_{\mu}\right),
\end{equation*}
as are the photon and \PZ boson,
\begin{equation*}
\left(\begin{array}{c}A_{\mu} \\ Z_{\mu} \end{array}\right ) =
  \left ( \begin{array}{cc} \cos\thetaw & \sin\thetaw \\ -\sin\thetaw &
      \cos\thetaw\end{array}\right)
\left ( \begin{array}{c} B_{\mu} \\ W^3_{\mu} \end{array} \right ),
\end{equation*}
where $\thetaw$ is known as the Weinberg angle. It is related to the coupling
constants $g$ and $g'$ by
\begin{eqnarray}
\sin\thetaw = \frac{g'}{\sqrt{g + {g'}^2}} \\
\cos\thetaw = \frac{g}{\sqrt{g + {g'}^2}}.
\end{eqnarray}

The $\Uone_{\textrm{EM}}$ symmetry describing electromagnetism is now formed
from a superposition of generators in the \SUtwo and $\Uone_{\textrm{Y}}$
groups. This model now includes the correct charge assignments to the gauge
bosons. Parity violation in the weak sector has also been included.

\subsection{Remaining Issues}\label{sec:remaining_issues}
The above model is now remarkably close to the full electroweak
theory. Unfortunately, two major problems remain -- both of them relating to
mass. Firstly, the leptons do not yet have mass. Naively, one might be tempted to
add a mass term with the following form:
\begin{equation*}
m\APsiL\PsiL = m(\APneutrino_{\textrm{L}}\Pneutrino_{L} +
\APlepton_{\textrm{L}}\Plepton_{\textrm{L}}).
\end{equation*}
However, terms of this form are required to vanish. This can be seen by applying
left-handed and right-handed projection operators to the spinors. The second
issue relates to the mass of the gauge bosons -- in particular how to generate
masses for the \PWp, \PWm and \PZ bosons whilst leaving the photon
massless. Both issues will be addressed in the next section.

\section{\acl{EWSB}}
\label{sec:theory_ewsb}
In order to give mass to the weak gauge bosons, as well as other fermions in the
\ac{SM}), a mechanism known as \acf{EWSB} is employed~\cite{ewsb_intro}. This
posits that, although the actual Lagrangian is invariant under a certain
symmetry group, the vacuum state of the theory is not.

\subsection{A Real Scalar Field}
To illustrate \ac{EWSB}, we will return to a simplified model with a single real
scalar field
\begin{equation*}
\lagrangiand = \left(\partial^{\mu}\phi\right)\left(\partial_{\mu}\phi\right)
-V(\phi),
\end{equation*}
where $V$ is the potential,
\begin{equation*}
V(\phi) = \frac{1}{2}\mu^2\phi^2 + \frac{1}{4}\lambda\phi^4.
\end{equation*}
The lowest energy states, $\phi_0$, are spacetime independent ($\partial_{\mu}
\phi_0 = 0$) and minimising $V$ we find,
\begin{equation*}
\phi_0 \left (\mu^2 + \lambda\phi_0^2\right) = 0.
\end{equation*}
To ensure that total energy is bounded below, $\lambda$ should be positive. For
$\mu^2 > 0$, there is one solution $\phi_0 = 0$. For $\mu^2 < 0$, there are two
solutions $\phi^\pm_0 = \pm \sqrt{-\mu^2/\lambda}$. Recall that the initial
Lagrangian is invariant under the transformation $\phi \longrightarrow
-\phi$. Given that one of the two possible vacua must be chosen, the vacuum is
no longer invariant under this symmetry. Furthermore, it is possible to expand
around the new vacuum (i.e. either $\phi_0^+$ or $\phi_0^-$),
$v=\sqrt{-\mu^2/\lambda}$ and define a new field $\phi'$ such that
\begin{equation*}
\phi' \equiv \phi - v.
\end{equation*}
The Lagrangian becomes
\begin{equation*}
\lagrangiand = \frac{1}{2}
\left(\partial_{\mu}\phi'\right)\left(\partial^{\mu}\phi'\right)
-\frac{1}{2}\left(\sqrt{-2\mu^2}\right)^2{\phi'}^2 - \lambda v {\phi'}^3
-\frac{1}{4}\lambda{\phi'}^4,
\end{equation*}
which, with the addition of the $\phi^3$ term, no longer respects the $\phi
\longrightarrow -\phi$ symmetry.

\subsection{A Complex Scalar Field and Goldstone's Theorem}
\label{sec:sm_goldstone}
Moving now to the case of a complex scalar field,
\begin{equation*}
\lagrangiand = \left(\partial_{\mu}\phi\right)\left(\partial^{\mu}\phi^*\right)
- V\left(\phi^*\phi\right),
\end{equation*}
where the potential is written
\begin{equation*}
V\left(\phi^*\phi\right) = \mu^2\left(\phi^*\phi\right) +
\lambda\left(\phi*\phi\right)^2,
\end{equation*}
the vacua are now $\left|\phi_0\right|^2 = -\mu^2/2\lambda$. The vacuum is now
symmetric under a global \Uone symmetry (as is the original Lagrangian). Writing
\begin{equation*}
\phi = \frac{1}{\sqrt{2}}\left(\phi_1 + i\phi_2\right),
\end{equation*}
and picking a vacuum configuration $\phi_1 = v$, $\phi_2 = 0$ we can once again
expand around the new vacuum. This yields the Lagrangian,
\begin{equation*}
  \lagrangiand =
  \frac{1}{2}\left(\partial_{\mu}\phi'_1\right)\left(\partial^{\mu}\phi'_1\right)
  -\frac{1}{2}\left(-2\mu^2\right){\phi'_1}^2 +
  \frac{1}{2}\left(\partial_{\mu}\phi'_2\right)\left(\partial^{\mu}\phi'_2\right)
  + \textrm{(interaction terms)}.
\end{equation*}
We can now identify a massive scalar field $\phi_1$ and a massless scalar field
$\phi_2$. This is an example of Goldstone's theorem -- when an exact continuous
global symmetry is broken, a massless scalar field will appear for each broken
group generator~\cite{nambu, goldstone}. In this case, the original \Uone
symmetry of the group has been broken, resulting in a single massless Goldstone
boson.

\subsection{The Higgs Mechanism}\label{sec:sm_higgs}
Massless scalar Goldstone bosons are theoretically undesirable as they have not
been observed by experiment. Fortunately, the Higgs
mechanism~\cite{higgs,kibble,englert} offers a solution to this problem, while
also giving the necessary mass to the weak gauge bosons. This is accomplished by
extending the global symmetry shown above to a local one.

Consider again the case of \ac{SQED} (\eqn~\ref{eqn:scalar_qed}). For small
perturbations, the fields may be expanded around the vacuum as follows,
\begin{equation*}
\phi = \exp\left(i\frac{\phi'_2}{v}\right)\frac{1}{\sqrt{2}}\left(\phi'_1 + v\right) \approx
\phi' + \frac{v}{\sqrt{2}}.
\end{equation*}
When substituted into the Lagrangian, this gives
\begin{align}
\label{eqn:higgs_lag}
  \lagrangiand =
  \frac{1}{2}\left(\partial_{\mu}\phi'_1\right)\left(\partial^{\mu}\phi'_1\right)
  - \frac{1}{2}\left(-2\mu^2\right){\phi'_1}^2 +
  \frac{1}{2}\left(\partial_{\mu}\phi'_2\right)\left(\partial^{\mu}\phi'_2\right)
  + \textrm{interaction terms}\nonumber\\
  - \frac{1}{4}F_{\mu\nu}^{\mu\nu} +
  \frac{q^2v^2}{2}gvA_{\mu}\left(\partial^{\mu}\phi'_2\right).
\end{align}
Two thing should be noted about the resulting Lagrangian. Firstly, that it
possesses a scalar field with mass $\sqrt{-2\mu^2}$. Secondly that the vector
boson $A_{\mu}$ has acquired a mass, $gv$. The last term in
\eqn~\ref{eqn:higgs_lag} is problematic, but can be removed by moving to the
``unitary gauge''~\cite{peskin_schroeder}.

It can be seen that the Goldstone boson has been absorbed into the vector boson,
adding an additional degree of freedom and causing it to acquire mass. This is
the essence of the mechanism employed in the \ac{SM} to give mass to the weak
gauge bosons whilst leaving the photon massless. The
$\SUtwo_{\textrm{L}}\times\Uone_{Y}$ symmetry of the \ac{SM} is broken down to
the $\Uone_{\textrm{EM}}$ symmetry of electromagnetism. Goldstone bosons for
each broken generator are absorbed by the \PWp, \PWm and \PZ -- giving them
mass. The symmetry corresponding to electromagnetism remains unbroken, and hence
the photon remains massless.

It should be noted that the Higgs mechanism also predicts the existence of a
massive scalar (a spin$-0$ particle) -- the $\phi$ field in the Lagrangian
above. This is the famous Higgs particle which has been the focus of many recent
particle physics experiments -- including the \ac{LHC}.

\subsection{Yukawa Couplings}
\label{sec:theory_yukawa}
It was noted in \sec~\ref{sec:remaining_issues} that it is not possible to write
down a straight forward mass term for the fermions in the electroweak theory. It
turns out that this problem is also solved by the Higgs mechanism by the
addition of a ``Yukawa coupling'' to the Lagrangian~\cite{peskin_schroeder},
\begin{equation*}
\lagrangiand = g \left(\APsiL \phi \PsiR + \APsiR\phi^{\dagger}\PsiL\right),
\end{equation*}
where $\phi$ is the Higgs field. When $\phi$ acquires a vacuum expectation value
via \ac{EWSB}, it can be seen that the fermion field $\Psi$ acquires a mass like
term -- with
\begin{equation*}
m_{\Psi} = \frac{gv}{\sqrt{2}}.
\end{equation*}
It is thought that all \ac{SM} fermions acquire mass in this way.

\section{\acl{QCD}}
\label{sec:sm_qcd}
The preceding discussion has covered only the electroweak sector of the \ac{SM}
- leptons, electroweak gauge bosons and the Higgs particle. The strong force is
incorporated within the framework of \ac{QCD}~\cite{pink_book}.  Whilst this at first appears to
be quite straightforward, certain properties of the strong force often make
calculations considerably more difficult. Here, a brief summary of the theory
will be given with further details available in~\cite{pink_book}.

\subsection{Quarks}
The quarks can be included in the \ac{SM} in a similar manner to the
leptons. The left-handed components of up and down type are paired into doublets,
\begin{equation*}
Q=\left(\begin{array}{c}\Pup_{\textrm{L}} \\ \Pdown_{\textrm{L}}\end{array}\right).
\end{equation*}
There is one such doublet for each of the three generations. The right-handed
components are once again placed in a singlet representation.

The weak eigenstates of the quarks are rotated with respect to their mass
eigenstates via the \acl{CKM} matrix~\cite{cabibbo,kobayashi_maskawa}. This
encodes the strength of various flavour-changing weak decays.

\subsection{Colour}
Inclusion of \ac{QCD} in the \ac{SM} continues in the same manner as for the
electroweak force. The local gauge symmetry group is extended to include an
additional product, $\SUthree_{C}$ where the $C$ indicates a new conserved
quantity -- \emph{colour charge}. The quarks -- of which six flavours are
currently known -- each carry one of three colour charges. These are labelled
\emph{red}, \emph{green} and \emph{blue}. As before, local invariance under this
transformation group requires the introduction of a covariant derivative and a
set of 8 gauge bosons -- the gluons.

\subsection{Unique Properties}
In the case of the weak force, it was seen that the non-Abelian generators of
the \SUtwo group give rise to self-interactions in the gauge bosons. The same is
true for the \SUthree gauge group of \ac{QCD}. However, in this case the
situation is further complicated by the energy-scaling behaviour of the strong
coupling constant.

At large energies -- or equivalently, short distances -- the strong coupling
constant becomes relatively small~\cite{lattice_qcd} and perturbation theory can
be used reliably. This is a phenomenon known as \emph{asymptotic
  freedom}~\cite{asymptotic_freedom}. At lower energies (or long distances), the
coupling constant becomes much larger. This causes the theory to become highly
non-linear~\cite{lattice_qcd2} and standard perturbation theory can no longer be
used.

An additional phenomenon, not predicted by perturbation
theory~\cite{perturbative_qcd_handbook} is \emph{confinement}. Confinement
forbids the existence of free quarks, forcing them to exist only as bound states
- baryons or mesons. This observation has been repeatedly confirmed by a number
of experimental searches~\cite[p.~30]{pdg}.

Many additional techniques have been developed to tackle these difficulties, for
instance Lattice \ac{QCD}~\cite{lattice_qcd}. In spite of this, \ac{QCD}
processes are often poorly understood, particularly at low energies. This has
important consequences for high energy physics experiments and will be an
important consideration in subsequent chapters.

Having covered necessary material on the \ac{SM}, it can be seen that the
\ac{SM} appears to be enormously successful in explaining the fundamental
particles and forces so far observed in experiments. Seemingly, the only
remaining issue is the discovery of the Higgs boson, which has not yet been
observed experimentally. However, even with the Higgs boson discovered, a number
of additional problems remain. These will be explored in the next chapter.
