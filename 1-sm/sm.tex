\chapter{The Standard Model}
\label{sec:sm}
\section{Introduction}
The \acl{SM} of particle physics is the best available theory describing the
interactions of all known fundamental particles. It is perhaps the most
extensively tested of any fundamental theory of nature, having withstood the
scrutiny of several decades of data-taking at a number of large, high-energy
physics experiments. However, despite this success, there appear to be a number
of major theoretical deficiencies in desperate need of attention. In this
chapter, the foundations of the theory will be laid out in a concise manner. The
aforementioned theoretical difficulties will be described in detail. This will
serve to motivate the potential solutions offered by the following chapter, and
ultimately the analysis work that has been undertaken.

\section{Particles and Fundamental Forces}
The \ac{SM} represents the unification of three of the four known fundamental
forces, namely: electromagnetism, the weak nuclear force and the strong nuclear
force. The other force, gravity, having resisted unification with the other
three, is not part of the \ac{SM}.

Within the \ac{SM}, each fundamental force is mediated by one or more ``gauge
bosons''. For electromagnetism, this is the photon, for the weak force, the
\PWp, \PWm and \PZ. Strong interactions are believed to be mediated by a set of
8 ``gluons'' (\Pg). Each boson is a vector particle and thus has a spin of 1.

In addition to the bosons, the \ac{SM} must of course catalogue the many matter
particles discovered by experiment. These are the fermions and may be further
subdivided into leptons and quarks. The leptons, or ``light particles'' may be
divided into three generations. Each generation associates a relatively heavy
charged lepton with a much lighter neutral partner, referred to as a
neutrino. The charged leptons are: the electron (\Pe), the muon (\Pgm) and the
tau lepton (\Ptau). Each has the same charge, typically written in terms of the
charge carried by a single electron, \ec. The corresponding neutrinos are then
referred to as the electron neutrino (\Pnue), muon neutrino (\Pnum) and tau
neutrino (\Pnut) respectively.

The quarks also appear to occupy three generations, with two quarks occupying
each. The first of each generation is referred to as an ``up-type'' quark and
has charge $+\frac{2}{3}\ec$, the second ``down-type'' with charge
$-\frac{1}{3}\ec$. The up-type quarks are (in order of increasing mass): up
(\Pup), charm (\Pcharm) and top (\Ptop). Similarly, the respective down-type
partners are: down (\Pdown), strange (\Pstrange) and bottom\footnote{also known
  as beauty} (\Pbottom). In addition to electromagnetism and the weak force,
quarks interact via the strong nuclear force.

Each of the fermions is associated with an anti-particle partner, with opposite
charge. For the charged leptons, the charge may be indicated by a superscript,
e.g. \Pep, \Pem, \Pgmp, \Pgmm etc. For the quarks, the anti-particles will
normally be denoted \APup, \APdown etc. In the case of the neutrinos, being
electromagnetically neutral, the question of the nature of the anti-particles is
as yet unanswered. It is theoretically possible for a neutrino to be its own
anti-particle, $\Pnulepton = \APnulepton$. This would make the neutrino a
``Majorana Fermion'' and is an area of active experimental research.

The particle content and properties of the \ac{SM} seem, at first glance, to be
quite arbitrary. Why are there three forces? Why does the Weak sector have three
bosons, and the electromagnetic only one. Some of these questions may be
answered by constructing the \ac{SM} as a gauge theory. This will be outlined in
the next section.

\section{Electroweak Gauge Theory}
\subsection{Gauge Invariance}
The theoretical principle of gauge invariance can be seen in Maxwell's theory of
electromagnetism \cite{aitchison}. Recall that the electric (\emE) and magnetic (\emB) components
of the field may be written in terms of a vector potential (\emA) and a scalar
potential \emV as follows:
\begin{eqnarray}
\emB = \nablav \times \emA \\
\emE = -\nablav \emV - \frac{\partial \emA}{\partial t}
\end{eqnarray}
It can be seen that these equations do not relate a given (\emB, \emE) to unique
values of \emA and \emV. In particular, if the following transformations are
applied simulatenously to \emA and \emV, the values of \emB and \emE will be
unchanged,
\begin{eqnarray}
\emA \longrightarrow \emA - \nablav \chi \\
\emV \longrightarrow  \emV + \frac{\partial \chi}{\partial t}
\end{eqnarray}
where $\chi$ is some arbitrary function. Such transformations are known as
``gauge transformations'' and similarly \emB and \emE are said to be ``gauge
invariant''.

If the scalar and vector potentials are rewritten in terms of a single 4-vector
potential, $\Amu = (\emV, \emA)$ and similarly taking a 4-vector differntial
operator, $\dmu = (\partial/\partial t, -\nablav)$, the gauge transformation
takes the form
\begin{equation}
\Amu \longrightarrow \Amu + \dmu\chi
\label{eqn:theory_gauge_transform}
\end{equation}
In this form, the Maxwell equations can be rewritten as
\begin{equation}
\dmu \Fmunu = j^{\nu}_{\textrm{em}}
\end{equation}
where $\Fmunu$ is the electromagnetic tensor, defined as
\begin{equation}
\Fmunu = \dmu\Anu - \dnu\Amu
\end{equation}
It can be seen that under the gauge transformation
(Eqn~\ref{eqn:theory_gauge_transform}), $\Fmunu$ is invariant. This confirms the
gauge invariance of the Maxwell equations.

\subsection{The Principle of Least Action and Lagrangian Formalism}
Before continuing, it is useful to review the Lagrangian formalism and the
principle of least action. The action, \action, is a quantity associated with a
physical system, used to characterise its dynamics. The action can be written
\begin{equation}
  S = \int dt \lagrangian = \int d^4 x \lagrangiand
\end{equation}
where \lagrangian is the \emph{Lagrangian}, and \lagrangiand the
\emph{Lagrangian density} and
\begin{equation}
\lagrangian = \int d^3 x \lagrangiand
\end{equation}
The Lagrangian is a function of some generalized coordinates $f_i$ describing
the dynamics of the system and their derivatives $\dmu f_i$. From this, the
Euler-Lagrange relation can be used to derive an equation of motion for the
system,
\begin{equation}
\frac{\partial \lagrangiand}{\partial f_i} - \partial_{\mu} \left (
  \frac{\partial \lagrangiand}{\partial\left(\partial_{\mu} f_i\right)}\right) = 0
\end{equation}
By writing down such a Lagrangian, a theory can be fully described. In order to
perform calculation of real measurable quantities, the theory must be
quantised. This is a complex procedure involving \emph{renormalisation}
techniques which are beyond the scope of this discussion. We shall only note
that renormalisability requires a theory with dimensionality at most $[M]^4$
i.e. mass to the fourth power.

\subsection{A Real Scalar Field}
To see how a theory can be written in terms of a Lagrangian, we will first
approach a highly simplified example, namely that of a \emph{real scalar
  field}. The Lagrangian density for such a theory may be written as follows
\begin{equation}
\lagrangiand =
\frac{1}{2}\left(\partial_{\mu}\phi(x))(\partial^{\mu}\phi(x)\right) -
\frac{1}{2}m^2\phi^2(x) - \frac{\lambda}{4!}\phi^4(x)
\label{eqn:theory_phi4}
\end{equation}
The three components of this expression are known respectively as the
\emph{kinetic term}, the \emph{mass term} and the \emph{interaction term}. The
field $\phi$ is a function of spacetime coordinates $x$ representing a single
kind of particle $\phi$, $m$ is the mass of the particle and $\lambda$ a
constant controlling the strength of the interaction between particles in the
theory.

Setting $\lambda = 0$ and applying the Euler-Lagrange relations
to \label{eqn:theory_phi4}, one obtains
\begin{equation}
\partial^{\mu}\partial_{\mu} \phi + m^2\phi = 0
\end{equation}
which can be recognised as the Klein-Gordon equation describing the motion of a
scalar particle.

\subsection{Symmetries}
It can be seen that the Lagrangian of Equation~\ref{eqn:theory_phi4} is
invariant under certain transformations. In particular, it has been constructed
to obey the symmetry $\phi \longrightarrow -\phi$. Such symmetries are known as
global symmetries of the theory - global in the sense that they are performed
identically at all points in spacetime. These symmetries can be classified in
terms of the mathematical theory of groups.

Noether's theorem states that for a theory with a continuous symmetry, each
generator of the corresponding symmetry group corresponds to a conserved current
in the theory. This allows predictions about the physics of given theory to be
made by considering only its symmetries.

\subsection{Complex Scalar Fields and the Gauge Principle}
We will now take a slightly more realistic example, that of a complex scalar
field. The Lagrangian for this theory may be written as follows,
\begin{equation}
\lagrangiand =
\frac{1}{2}\left(\partial_{\mu}\phi(x))(\partial^{\mu}\phi(x)\right)^* -
\frac{1}{2}m^2\left|\phi(x)\right|^2 - \frac{\lambda}{4!}\left(\left|\phi(x)\right|^2\right)
\label{eqn:theory_complex_scalar}
\end{equation}
where the field $\phi$ is now a complex quantity. This field is seen to possess
an additional symmetry
\begin{equation}
\phi \longrightarrow e^{i\theta}\phi
\label{eqn:theory_phase_transform}
\end{equation}
where $\theta$ is some arbitrary parameter (constant, for the moment, in
spacetime). This is known as a global \Uone symmetry and has a single generator
- and therefore also a single conserved current.

When the transformation of Eqn~\ref{eqn:theory_phase_transform} is extended such
that the parameter $\theta$ is a function of the spacetime coordinate $x$, the
Lagrangian is no longer invariant,
\begin{equation}
\delta\lagrangiand = \left(\partial_{\mu}
  \theta\right) \left[i\left(\partial^{\mu} \phi^*\right)\phi
  -i\phi^*\left(\partial^{\mu} \phi\right)\right] +
\left(\partial_{\mu}\theta\right)\left(\partial^{\mu}\theta\right)\left|\phi\right|^2
\label{eqn:theory_gauge_dL}
\end{equation}
In order to compensate for this change, we define the covariant derivative as follows
\begin{equation}
D_{\mu} = \partial_{\mu} - iB_{\mu}(x)
\label{eqn:theory_cov_deriv}
\end{equation}
where $B$ is taken to be a new four-vector field. This operator must transform
under a change of phase in such a way as to cancel the additional terms in
Equation~\ref{eqn:theory_gauge_dL}. The Lagrangian would then be rewritten as
follows,
\begin{equation}
  \lagrangiand = \left(D_{\mu}\phi\right)\left(D^{\mu}\phi\right)^*
  - \frac{1}{2}m^2\left|\phi(x)\right|^2 - \frac{\lambda}{4!}\left(\left|\phi(x)\right|^2\right)
\end{equation}
It can be shown that this requires the $B_{\mu}$ field to transform as follows
\begin{equation}
B_{\mu} \longrightarrow B'_{\mu} = B_{\mu} + \partial_{\mu}\theta
\end{equation}
It can be seen that this is the same transformation as that applied to the
electromagnetic four-vector field in
Equation~\ref{eqn:theory_gauge_transform}. By requiring the fields to be
symmetric under a \Uone phase transformation, a new field must be
introduced. This field then appears as a gauge boson in the theory, in this case
a photon.

Whilst this might seem a fairly trivial theoretical exercise, it turns out that
by repeating the same procedure with different symmetry groups, all gauge bosons
of the \ac{SM} can be predicted.




\section{The Higgs Mechanism}
\section{\acl{QCD}}
