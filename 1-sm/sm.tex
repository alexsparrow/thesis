\chapter{The Standard Model}
\label{sec:sm}
\section{Introduction}
The \acl{SM} of particle physics is the best available theory describing the
interactions of all known fundamental particles. It is perhaps the most
extensively tested of any fundamental theory of nature, having withstood the
scrutiny of several decades of data-taking at a number of large, high-energy
physics experiments. However, despite this success, there appear to be a number
of major theoretical deficiencies in desperate need of attention. In this
chapter, the foundations of the theory will be laid out in a concise
manner. The aforementioned theoretical difficulties will be described in
detail. This will sever to motivate the potential solutions offered by the
following chapter, and ultimately the analysis work that has been undertaken.

\section{Particles and Fundamental Forces}
The \ac{SM} represents the unification of three of the four known fundamental
forces, namely: electromagnetism, the weak nuclear force and the strong nuclear
force. The other force, gravity, having resisted unification with the other
three, is not part of the \ac{SM}.

Within the \ac{SM}, each fundamental force is mediated by one or more ``gauge
bosons''. For electromagnetism, this is the photon, for the weak force, the
\PWp, \PWm and \PZ. Strong interactions are believed to be mediated by a set of
8 ``gluons'' (\Pg). Each boson is a vector particle and thus has a spin of 1.

In addition to the bosons, the \ac{SM} must of course catalogue the many matter
particles discovered by experiment. These may be further subdivided into leptons
and quarks. The leptons, or ``light particles'' may be divided into three
generations. Each generation associates a relatively heavy charged lepton with a
much lighter neutral partner, referred to as a neutrino. The charged leptons
are: the electron (\Pe), the muon (\Pgm) and the tau lepton (\Ptau). Each has the
same charge, typically written in terms of the charge carried by a single
electron, \ec. The corresponding neutrinos are then referred to as the electron
neutrino (\Pnue), muon neutrino (\Pnum) and tau neutrino (\Pnut) respectively.

The quarks also appear to occupy three generations, with two quarks occupying
each. The first of each generation is referred to as an ``up-type'' quark and
has charge $+\frac{2}{3}\ec$, the second ``down-type'' with charge
$-\frac{1}{3}\ec$. The up-type quarks are (in order of increasing mass): up
(\Pup), charm (\Pcharm) and top (\Ptop). Similarly, the respective down-type
partners are: down (\Pdown), strange (\Pstrange) and bottom\footnote{also known
  as beauty} (\Pbottom). In addition to electromagnetism and the weak force,
quarks interact via the strong nuclear force.



\section{Electroweak Gauge Theory}
\section{The Higgs Mechanism}
\section{\acl{QCD}}
