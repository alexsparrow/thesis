\chapter{The Standard Model}
\label{sec:sm}
\section{Introduction}
The \acl{SM} of particle physics is the best available theory describing the
interactions of all known fundamental particles. It is perhaps the most
extensively tested of any fundamental theory of nature, having withstood the
scrutiny of several decades of data-taking at a number of large, high-energy
physics experiments. However, despite this success, there appear to be a number
of major theoretical deficiencies in desperate need of attention. In this
chapter, the foundations of the theory will be laid out in a concise manner. The
aforementioned theoretical difficulties will be described in detail. This will
serve to motivate the potential solutions offered by the following chapter, and
ultimately the analysis work that has been undertaken.

It is important to note that, although the \ac{SM} will be presented here as a
complete theory, seemingly designed to match the currently observed set of
fundamental particles, this is not how it came into being. The theory was built
up over much of the second half of the 20th centry and actually successfully
predicited a number of discoveries. Perhaps the best example of this is the
discovery of the \PW and \PZ bosons by the \ac{UA1} and \ac{UA2} experiments at
\ac{CERN} in 1983. These had been theorised by Weinberg, Glashow and Salam in
1968.

\section{Particles and Fundamental Forces}
\label{sec:theory:particles}
The \ac{SM} represents the unification of three of the four known fundamental
forces, namely: electromagnetism, the weak nuclear force and the strong nuclear
force. The other force, gravity, having resisted unification with the other
three, is not part of the \ac{SM}.

Within the \ac{SM}, each fundamental force is mediated by one or more ``gauge
bosons''. For electromagnetism, this is the photon, for the weak force, the
\PWp, \PWm and \PZ. Strong interactions are believed to be mediated by a set of
8 ``gluons'' (\Pg). Each boson is a vector particle and thus has a spin of 1.

In addition to the bosons, the \ac{SM} must of course catalogue the many matter
particles discovered by experiment. These are the fermions and may be further
subdivided into leptons and quarks. The leptons, or ``light particles'' may be
divided into three generations. Each generation associates a relatively heavy
charged lepton with a much lighter neutral partner, referred to as a
neutrino. The charged leptons are: the electron (\Pe), the muon (\Pgm) and the
tau lepton (\Ptau). Each has the same charge, typically written in terms of the
charge carried by a single electron, \ec. The corresponding neutrinos are then
referred to as the electron neutrino (\Pnue), muon neutrino (\Pnum) and tau
neutrino (\Pnut) respectively.

The quarks also appear to occupy three generations, with two quarks occupying
each. The first of each generation is referred to as an ``up-type'' quark and
has charge $+\frac{2}{3}\ec$, the second ``down-type'' with charge
$-\frac{1}{3}\ec$. The up-type quarks are (in order of increasing mass): up
(\Pup), charm (\Pcharm) and top (\Ptop). Similarly, the respective down-type
partners are: down (\Pdown), strange (\Pstrange) and bottom\footnote{also known
  as beauty} (\Pbottom). In addition to electromagnetism and the weak force,
quarks interact via the strong nuclear force.

Each of the fermions is associated with an anti-particle partner, with opposite
charge. For the charged leptons, the charge may be indicated by a superscript,
e.g. \Pep, \Pem, \Pgmp, \Pgmm etc. For the quarks, the anti-particles will
normally be denoted \APup, \APdown etc. In the case of the neutrinos, being
electromagnetically neutral, the question of the nature of the anti-particles is
as yet unanswered. It is theoretically possible for a neutrino to be its own
anti-particle, $\Pnulepton = \APnulepton$. This would make the neutrino a
``Majorana Fermion'' and is an area of active experimental research.

The particle content and properties of the \ac{SM} seem, at first glance, to be
quite arbitrary. Why are there three forces? Why does the Weak sector have three
bosons, and the electromagnetic only one. Some of these questions may be
answered by constructing the \ac{SM} as a gauge theory. This will be outlined in
the next section.
\section{Electroweak Gauge Theory}
\subsection{Gauge Invariance}
The theoretical principle of gauge invariance can be seen in Maxwell's theory of
electromagnetism \cite{aitchison}. Recall that the electric (\emE) and magnetic (\emB) components
of the field may be written in terms of a vector potential (\emA) and a scalar
potential \emV as follows:
\begin{eqnarray}
\emB = \nablav \times \emA \\
\emE = -\nablav \emV - \frac{\partial \emA}{\partial t}
\end{eqnarray}
It can be seen that these equations do not relate a given (\emB, \emE) to unique
values of \emA and \emV. In particular, if the following transformations are
applied simulatenously to \emA and \emV, the values of \emB and \emE will be
unchanged,
\begin{eqnarray}
\emA \longrightarrow \emA + \nablav \chi \\
\emV \longrightarrow  \emV - \frac{\partial \chi}{\partial t}
\end{eqnarray}
where $\chi$ is some arbitrary function. Such transformations are known as
``gauge transformations'' and similarly \emB and \emE are said to be ``gauge
invariant''.

If the scalar and vector potentials are rewritten in terms of a single 4-vector
potential, $\Amu = (\emV, \emA)$ and similarly taking a 4-vector differntial
operator, $\dmu = (\partial/\partial t, -\nablav)$, the gauge transformation
takes the form
\begin{equation}
\Amu \longrightarrow \Amu - \dmu\chi
\label{eqn:theory_gauge_transform}
\end{equation}
In this form, the Maxwell equations can be rewritten as
\begin{equation}
\dmu \Fmunu = j^{\nu}_{\textrm{em}}
\end{equation}
where $\Fmunu$ is the electromagnetic tensor, defined as
\begin{equation}
\Fmunu = \dmu\Anu - \dnu\Amu
\end{equation}
It can be seen that under the gauge transformation
(Eqn~\ref{eqn:theory_gauge_transform}), $\Fmunu$ is invariant. This confirms the
gauge invariance of the Maxwell equations.

\subsection{The Principle of Least Action and Lagrangian Formalism}
Before continuing, it is useful to review the Lagrangian formalism and the
principle of least action. The action, \action, is a quantity associated with a
physical system, used to characterise its dynamics. The action can be written
\begin{equation}
  S = \int dt \lagrangian = \int d^4 x \lagrangiand
\end{equation}
where \lagrangian is the \emph{Lagrangian}, and \lagrangiand the
\emph{Lagrangian density} and
\begin{equation}
\lagrangian = \int d^3 x \lagrangiand
\end{equation}
The Lagrangian is a function of some generalized coordinates $f_i$ describing
the dynamics of the system and their derivatives $\dmu f_i$. From this, the
Euler-Lagrange relation can be used to derive an equation of motion for the
system,
\begin{equation}
\frac{\partial \lagrangiand}{\partial f_i} - \partial_{\mu} \left (
  \frac{\partial \lagrangiand}{\partial\left(\partial_{\mu} f_i\right)}\right) = 0
\end{equation}
By writing down such a Lagrangian, a theory can be fully described. In order to
perform calculation of real measurable quantities, the theory must be
quantised. This is a complex procedure involving \emph{renormalisation}
techniques which are beyond the scope of this discussion. We shall only note
that renormalisability requires a theory with dimensionality at most $[M]^4$
i.e. mass to the fourth power.

\subsection{A Real Scalar Field}
To see how a theory can be written in terms of a Lagrangian, we will first
approach a highly simplified example, namely that of a \emph{real scalar
  field}. The Lagrangian density for such a theory may be written as follows
\begin{equation}
\lagrangiand =
\frac{1}{2}\left(\partial_{\mu}\phi(x))(\partial^{\mu}\phi(x)\right) -
\frac{1}{2}m^2\phi^2(x) - \frac{\lambda}{4!}\phi^4(x)
\label{eqn:theory_phi4}
\end{equation}
The three components of this expression are known respectively as the
\emph{kinetic term}, the \emph{mass term} and the \emph{interaction term}. The
field $\phi$ is a function of spacetime coordinates $x$ representing a single
kind of particle $\phi$, $m$ is the mass of the particle and $\lambda$ a
constant controlling the strength of the interaction between particles in the
theory.

Setting $\lambda = 0$ and applying the Euler-Lagrange relations
to \label{eqn:theory_phi4}, one obtains
\begin{equation}
\partial^{\mu}\partial_{\mu} \phi + m^2\phi = 0
\end{equation}
which can be recognised as the Klein-Gordon equation describing the motion of a
scalar particle.

\subsection{Symmetries}
It can be seen that the Lagrangian of Equation~\ref{eqn:theory_phi4} is
invariant under certain transformations. In particular, it has been constructed
to obey the symmetry $\phi \longrightarrow -\phi$. Such symmetries are known as
global symmetries of the theory - global in the sense that they are performed
identically at all points in spacetime. These symmetries can be classified in
terms of the mathematical theory of groups.

Noether's theorem states that for a theory with a continuous symmetry, each
generator of the corresponding symmetry group corresponds to a conserved current
in the theory. This allows predictions about the physics of given theory to be
made by considering only its symmetries.

\subsection{Complex Scalar Fields and the Gauge Principle}
We will now take a slightly more realistic example, that of a complex scalar
field. The Lagrangian for this theory may be written as follows,
\begin{equation}
\lagrangiand =
\frac{1}{2}\left(\partial_{\mu}\phi(x))(\partial^{\mu}\phi(x)\right)^* -
\frac{1}{2}m^2\left|\phi(x)\right|^2 - \frac{\lambda}{4!}\left(\left|\phi(x)\right|^2\right)
\label{eqn:theory_complex_scalar}
\end{equation}
where the field $\phi$ is now a complex quantity. This field is seen to possess
an additional symmetry
\begin{equation}
\phi \longrightarrow e^{i\theta}\phi
\label{eqn:theory_phase_transform}
\end{equation}
where $\theta$ is some arbitrary parameter (constant, for the moment, in
spacetime). This is known as a global \Uone symmetry and has a single generator
- and therefore also a single conserved current.

When the transformation of Eqn~\ref{eqn:theory_phase_transform} is extended such
that the parameter $\theta$ is a function of the spacetime coordinate $x$, the
Lagrangian is no longer invariant,
\begin{equation}
\delta\lagrangiand = \left(\partial_{\mu}
  \theta\right) \left[i\left(\partial^{\mu} \phi^*\right)\phi
  -i\phi^*\left(\partial^{\mu} \phi\right)\right] +
\left(\partial_{\mu}\theta\right)\left(\partial^{\mu}\theta\right)\left|\phi\right|^2
\label{eqn:theory_gauge_dL}
\end{equation}
In order to compensate for this change, we define the covariant derivative as follows
\begin{equation}
D_{\mu} = \partial_{\mu} + igB_{\mu}(x)
\label{eqn:theory_cov_deriv}
\end{equation}
where $B$ is taken to be a new four-vector field and $g$ is some constant. This
operator must transform under a change of phase in such a way as to cancel the
additional terms in Equation~\ref{eqn:theory_gauge_dL}. The Lagrangian would
then be rewritten as follows,
\begin{equation}
  \lagrangiand = \left(D_{\mu}\phi\right)\left(D^{\mu}\phi\right)^*
  - \frac{1}{2}m^2\left|\phi(x)\right|^2 - \frac{\lambda}{4!}\left(\left|\phi(x)\right|^2\right)
\end{equation}
It can be shown that this requires the $B_{\mu}$ field to transform as follows
\begin{equation}
B_{\mu} \longrightarrow B'_{\mu} = B_{\mu} - \frac{1}{g}\partial_{\mu}\theta
\end{equation}
It can be seen that this is the same transformation as that applied to the
electromagnetic four-vector field in
Equation~\ref{eqn:theory_gauge_transform}. By requiring the fields to be
symmetric under a \Uone phase transformation, a new field must be
introduced. This field then appears as a gauge boson in the theory, in this case
an analogue of the photon. Incorporating the Maxwell equations into the
Lagrangian, one arrives at \acl{SQED}.
\begin{equation}
  \lagrangiand = -\frac{1}{4}F_{\mu\nu}F^{\mu\nu} + \left(D_{\mu}\phi\right)\left(D^{\mu}\phi\right)^*
  - \frac{1}{2}m^2\left|\phi(x)\right|^2 -
  \frac{\lambda}{4!}\left(\left|\phi(x)\right|^2\right)
\label{eqn:scalar_qed}
\end{equation}

The procedure which has just been outlined is known as the \emph{gauge
  principle}. Whilst it might seem an interesting but trivial result, it turns
out that all particles and interactions of the \ac{SM} (see
Section~\ref{sec:theory:particles}) can be predicted by repeating this procedure
with larger symmetry groups.

\subsection{Yang-Mills Theory}
Having shown within a highly simplified model, how the gauge principle can be
used to predict the existence of gauge bosons by enforcing a local symmetry, we
shall now discuss the inclusion of additional gauge bosons by moving to a
``larger'' local symmetry group.

Extending the symmetry group to \SUtwo and replacing the single complex field
$\phi$ by a vector of complex fields $\Phi$,
\begin{eqnarray*}
\Phi \longrightarrow \Phi' = \exp{i\theta(x) + iT^a W^a_{\mu}}\Phi\\
D_{\mu} = \partial_{\mu} + igT^a_{\mu}
\end{eqnarray*}
where $T^a$ are the generators of the group \SUtwo, $T^a = \frac{1}{2}\sigma^a$
and the $\sigma^a$ are the well-known Pauli matrices. Notice that this symmetry
group predicts the existence of three gauge bosons - matching the three weak
gauge bosons: \PZ, \PWp and \PWm.

Another important aspect of this toy theory is related to the structure of the
symmetry group \SUtwo. Namely, that since the group is not Abelian - meaning
that its generators do not commute - the Lagrangian must be modified to ensure
local gauge invariance,
\begin{eqnarray*}
\lagrangiand = -\frac{1}{4}F^a_{\mu\nu}F^{\mu\nu} \\
F^a_{\mu\nu} = \partial_{\mu} A^a_{\nu} - \partial_{\nu} A^a_{\mu} + g f^{abc}
A^b_{\mu} A^c_{\nu}
\end{eqnarray*}
where $f^{abc}$ are known as structure constants. The additional quadratic terms
in the Lagrangian lead to self interactions, just as is observed in the weak
sector.

It would seem that a model based on a symmetry group $\SUtwo\times\Uone$ would
appear to give the correct number of degrees of freedom for a unification of the
electromagnetic and weak nuclear forces. One problem in particular is the fact
that the weak bosons are massive and the photon massless. It turns out that
giving the bosons mass directly in the Lagrangian (as for the field $\phi$
in \label{eqn:scalar_qed}) breaks gauge invariance. Another more subtle issue is
that the \PWp and \PWm would be uncharged in this picture, since the \SUtwo
component is able to commute with the \Uone part.

\subsection{Spin and Chirality}
Having arrived at a toy gauge theory bearing some similarity to the \ac{SM}, it
is important to introduce a concept not yet represented in the theory -
spin. The Lagrangians presented so far have worked only with scalar fields, but
these are known to not represent the physical world. In reality, particles have
spin - intrinsic angular momentum.

Fermionic degrees of freedom are represented as \emph{spinors}. These transform
in a different manner to vectors under spatial rotations. In addition, they obey
a different equation of motion, known as the \emph{Dirac Equation},
\begin{equation}
i\gamma^\mu \partial_{\mu}\Psi + m\Psi = 0
\end{equation}
where $\Psi$ is a spinor and $\gamma^{\mu}$ are known as the Diract matrices
(see e.g. \cite{aitchison}. The Dirac matrices obey the anti-commutation
relation $\{\gamma^{\mu}, \gamma^{\nu}\} = 2g^{\mu\nu}$ where $g^{\mu\nu}$ is
the spacetime metric. Writing an appropriate Lagrangian, one obtains,
\begin{equation}
\bar{\Psi} \left (i\gamma^{\mu}\partial_{\mu} -m\right)\Psi \quad \bar{\Psi} =
\Psi^{\dagger}\gamma_0
\end{equation}
One important aspect of spinors is a property known as ``handedness'' or
\emph{chirality}. Physically speaking, a left-handed particle is one whose spin
is aligned with its direction of motion and a right-handed one whose spin is
oppositely aligned. Note that a left-handed anti-spinor corresponds to a right
handed physical particle (see \cite{peskin_schroeder}).

\subsection{The Electroweak Theory}
A striking property of the electroweak theory is the observation of maximal
parity violation by Wu et al \cite{wu_parity}. By observing the $\beta$ decay of
Cobalt-60 atoms. The atoms had their spins polarised by a magnetic field and the
angular distribution of electrons was measured. It was seen that electrons were
emitted preferentially in the opposite direction to their spin. This implies
that the neutrino momentum in the decay is always aligned in an opposite to its
spin - i.e. the neutrino is left-handed. More generally, the weak gauge bosons
interact only with left-handed particles and right-handed anti-particles. This
is also referred to as the $(V-A)$ nature of the theory - since the Lagrangian
contains ``vector minus axial'' terms which lead to parity violation. Parity
violation will be of vital importance to the analysis presented in
Chapter~\ref{sec:wpol}.

It is not necessary for us to fully detail the electroweak sector of the \ac{SM}
Lagrangian. For the sake of completeness, an outline of the construction of the
theory will be given. Firstly, the symmetry group chosen is
$\SUtwo_{\textrm{L}}\times \Uone_{Y}$ where the subscript $L$ indicates the
coupling only to left-handed spinors and the $Y$ that the $\Uone$ group here is
not electromagnetism but ``hypercharge''. This is essential to overcoming the
issues with $\SUtwo\times\Uone$ already described.

The fermions themselves are placed into ``isospin'' doublets coupling a charged
lepton with a neutrino.
\begin{equation}
\PsiL = \left(\begin{array}{c} \Pgn_{\textrm{L}} \\
    \Plepton_{\textrm{L}} \end{array}\right)
\end{equation}
where $\Pgn_{\textrm{L}}$ and $\Plepton_{\textrm{L}}$ are left-handed spinor
fields representing a neutral and charged lepton respectively. The right-handed
component of the charged lepton (which does interact electromagnetically) is
incorporated as an isospin singlet
\begin{equation}
\PsiR = \Plepton_{\textrm{R}}
\end{equation}


Considering the Lagrangian as
\begin{equation}
\lagrangiand = \lagrangiand_{\textrm{gauge}} + \lagrangiand_{\textrm{free}}
\end{equation}
and writing the analogue of the tensor \Fmunu for the gauge group
$\SUtwo_{\textrm{L}}\times\Uone$ $\SUtwo\times\Uone$,
\begin{equation}
\lagrangiand_{\textrm{gauge}} = -\frac{1}{4} W^{a\mu\nu} W^a_{\mu\nu}
-\frac{1}{4} B^{\mu\nu}B_{\mu\nu}
\end{equation}
where
\begin{eqnarray*}
W^{a}_{\mu\nu} = \partial_{\mu} W^a_{\nu} - \partial_{\nu}W^a_{\mu} + g f^{abc}
W^{b}_{\mu} W^c_{\nu}\\
B_{\mu\nu} = \partial_{\mu} B_{\nu} - \partial_{\nu} B_{\mu}
\end{eqnarray*}
The piece of the Lagrangian for free leptons is then,
\begin{equation}
\lagrangiand_{\textrm{leptons}} = \APsiL i \gamma^{\mu} D_{\mu} \PsiL + \APsiR i
\gamma^{\mu} D_{\mu} \PsiR
\end{equation}
The covariant derivative is written
\begin{equation}
D_{\mu} = \partial_{\mu} + i g W^a_{\mu} T^a + i g' B^{\mu}\frac{Y}{2}
\end{equation}
where the $T^a = \frac{1}{2}\sigma^a$ when acting on a left-handed spinor and
zero otherwise. Simiarly, the hypercharge $Y$ is $-1$ for left-handed spinors
and $-2$ for right-handed. The $g$ and $g'$ are coupling constants.

It can then be seen that the physical \PWp and \PWm are superpositions,
\begin{equation}
\PWpm_{\mu} = \frac{1}{\sqrt{2}}\left(\PW^1_{\mu} + \PW^2_{\mu}\right)
\end{equation}
and the photon and \PZ,
\begin{equation}
\left(\begin{array}{c}A_{\mu} \\ Z_{\mu} \end{array}\right ) =
  \left ( \begin{array}{cc} \cos\thetaw & \sin\thetaw \\ -\sin\thetaw &
      \cos\thetaw\end{array}\right)
\left ( \begin{array}{c} B_{\mu} \\ W^3_{\mu} \end{array} \right )
\end{equation}
where $\thetaw$ is known as the Weinberg angle. It is related to the coupling
constants $g$ and $g'$ by
\begin{eqnarray}
\sin\thetaw = \frac{g'}{\sqrt{g + {g'}^2}} \\
\cos\thetaw = \frac{g}{\sqrt{g + {g'}^2}}
\end{eqnarray}

So it can be seen that by introducing weak isospin and hypercharge, a more
consistent theory has been constructed. The $\Uone_{\textrm{EM}}$ symmetry
describing electomagnetism is now formed from a superposition of generators in
the \SUtwo and $\Uone_{\textrm{Y}}$ groups. This model now includes the correct
charge assignments to the gauge bosons as well as the parity violation observed
in the weak sector.

\subsection{Remaining Issues}
\label{sec:remaining_issues}
This model is now remarkably close to the full electroweak theory. Unfortunately
two major problems remain - both relating to mass. Firstly, the mass of the
leptons has not been included. Naively one would be tempted to add a mass term
\begin{equation}
m\APsiL\PsiL = m(\APneutrino_{\textrm{L}}\Pneutrino_{L} +
\APlepton_{\textrm{L}}\Plepton_{\textrm{L}})
\end{equation}
However, this term vanishes (this can be seen by applying left-handed and
right-handed projection operators to the spinors). The second issue relates to
the mass of the gauge bosons - in particular how to generate masses for the
\PWp, \PWm and \PZ whilst leaving the photon massless. Both issues will be
addressed in the next section.

\section{\acl{EWSB}}
In order to give mass to the weak gauge bosons (and other fermions in the
\ac{SM}), a mechanism called \acl{EWSB} is employed. This posits that although
the Lagrangian is invariant under some group of transformations, the vacuum
state of the theory ``breaks'' the symmetry down to a smaller group.

\subsection{A Real Scalar Field}
To illustrate how this works, we will return to a simplified model with a real
scalar field
\begin{equation}
\lagrangiand = \left(\partial^{\mu}\phi\right)\left(\partial_{\mu}\phi\right)
-V(\phi)
\end{equation}
where $V$ is the potential
\begin{equation}
V(\phi) = \frac{1}{2}\mu^2\phi^2 + \frac{1}{4}\lambda\phi^4
\end{equation}
The lowest energy states are spacetime independent ($\partial_{\mu} \phi_0 = 0$)
and mimising $V$ we find,
\begin{equation}
\phi_0 \left (\mu^2 + \lambda\phi_0^2\right) = 0
\end{equation}
To ensure that total energy is bounded below, $\lambda$ should be positive. For
$\mu^2 > 0$, there is one solution $\phi_0 = 0$. For $\mu^2 < 0$, there are two
solutions $\phi^\pm_0 = \pm \sqrt{-\mu^2/\lambda}$. Recall that the initial
Lagrangian is invariant under the transformation $\phi \longrightarrow
-\phi$. Given that one of the two possible vacua must be chosen, the vacuum is
no longer invariant under this symmetry. Furthermore, it is possible to expand
around the new vacuum (e.g. $\phi_0^\pm$), $v=\sqrt{-\mu^2/\lambda}$ and define
a new field $\phi'$,
\begin{equation}
\phi' \equiv \phi - v
\end{equation}
the Lagrangian becomes
\begin{equation}
\lagrangiand = \frac{1}{2}
\left(\partial_{\mu}\phi'\right)\left(\partial^{\mu}\phi'\right)
-\frac{1}{2}\left(\sqrt{-2\mu^2}\right)^2{\phi'}^2 - \lambda v {\phi'}^3
-\frac{1}{4}\lambda{\phi'}^4
\end{equation}
which, with the addition of the $\phi^3$ term, no longer respects the $\phi
\longrightarrow -\phi$ symmetry.

\subsection{A Complex Scalar Field and Goldstone's Theorem}
Moving now to the case of a complex scalar field,
\begin{equation}
\lagrangiand = \left(\partial_{\mu}\phi\right)\left(\partial^{\mu}\phi^*\right)
- V\left(\phi^*\phi\right)
\end{equation}
where the potential is written
\begin{equation}
V\left(\phi^*\phi\right) = \mu^2\left(\phi^*\phi\right) +
\lambda\left(\phi*\phi\right)^2
\end{equation}
the vacua are now $\left|\phi_0\right|^2 = -\mu^2/2\lambda$. The vacuum is now
symmetric under a global \Uone symmetry (as is the original Lagrangian). Writing
\begin{equation}
\phi = \frac{1}{\sqrt{2}}\left(\phi_1 + i\phi_2\right)
\end{equation}
and picking a vacuum configuration $\phi_1 = v$, $\phi_2 = 0$ we can once again
expand around the new vacuum. This yields the Lagrangian,
\begin{equation}
  \lagrangiand =
  \frac{1}{2}\left(\partial_{\mu}\phi'_1\right)\left(\partial^{\mu}\phi'_1\right)
  -\frac{1}{2}\left(-2\mu^2\right){\phi'_1}^2 +
  \frac{1}{2}\left(\partial_{\mu}\phi'_2\right)\left(\partial^{\mu}\phi'_2\right)
  + \textrm{(interaction terms)}
\end{equation}
we can now identify a massive scalar field $\phi_1$ and a massless scalar field
$\phi_2$. This is an example of Goldstone's theorem - when an exact continuous
global symmetry is broken, a massless scalar field will appear for each broken
group generator. In this case, the original \Uone symmetry of the group has been
broken, resulting in a single massless Goldstone boson.

\subsection{The Higgs Mechanism}
Massless scalar Goldstone bosons are theoretically undesirable as they have not
been observed by experiment. Fortunately, the Higgs mechanism\cite{higgs} offers
a solution to this problem while also giving the necessary mass to the weak
gauge bosons. This is accomplished by turning the global symmetry shown above
into a local one.

Consider again the case of \ac{SQED} (Eqn~\ref{eqn:scalar_qed}). For small
perturbations, the fields may be expanded around the vacuum as follows,
\begin{equation}
\phi = \exp\left(i\frac{\phi'_2}{v}\right)\frac{1}{\sqrt{2}}\left(\phi'_1 + v\right) \approx
\phi' + \frac{v}{\sqrt{2}}
\end{equation}
When substituted into the Lagrangian this gives
\begin{align}
  \lagrangiand =
  \frac{1}{2}\left(\partial_{\mu}\phi'_1\right)\left(\partial^{\mu}\phi'_1\right)
  - \frac{1}{2}\left(-2\mu^2\right){\phi'_1}^2 +
  \frac{1}{2}\left(\partial_{\mu}\phi'_2\right)\left(\partial^{\mu}\phi'_2\right)
  + \textrm{interaction terms}\\
  - \frac{1}{4}F_{\mu\nu}^{\mu\nu} +
  \frac{q^2v^2}{2}gvA_{\mu}\left(\partial^{\mu}\phi'_2\right)
\end{align}
Two thing should be noted about the resulting Lagrangian. Firstly, that it
posesses a scalar field with mass $\sqrt{-2\mu^2}$ and secondly a vector
boson $A_{\mu}$ with mass $gv$. The final term is problematic, but can be
easily removed by moving to the ``unitary gauge''.

It can be seen that the Goldstone boson has been absorbed into the vector boson,
adding an additional degree of freedom and allowing it to acquire mass. This is
the essence of the mechanism employed in the \ac{SM} to give mass to the weak
gauge bosons whilst leaving the photon massless. The
$\SUtwo_{\textrm{L}}\times\Uone_{Y}$ symmetry of the \ac{SM} is broken down to
the $\Uone_{\textrm{EM}}$ symmetry of electromagnetism. Goldstone bosons for
each broken generator are absorbed by the \PWp, \PWm and \PZ causing them to
acquire mass. The symmetry corresponding to electromagnetism remains unbroken,
leaving the photon massless.

It should be noted that, in addition to the massive vector bosons, the Higgs
mechanism predicts the existence of a massive scalar (a spin$-0$ particle) - the
$\phi$ field in the Lagrangian above. This is the famous Higgs particle which
has been the focus of extensive and continuing (as of this writing) experimental
searches.

\subsection{Yukawa Couplings}
It was noted in Section~\ref{sec:remaining_issues} that it is not possible to
write down a straight forward mass term for the fermions in the electroweak
theory. It turns out that this problem is also solved by the Higgs mecahnism by
the addition of a Yukawa coupling,
\begin{equation}
\lagrangiand = g \left(\APsiL \phi \PsiR + \APsiR\phi^{\dagger}\PsiL\right)
\end{equation}
where $\phi$ is the Higgs field. When $\phi$ acquires a vacuum expectation value
via \ac{EWSB}, it can be seen that the fermion field $\Psi$ acquires a mass like
term - with
\begin{equation}
m_{\Psi} = \frac{gv}{\sqrt{2}}
\end{equation}
It is thought that all \ac{SM} fermions acquire their mass in this way.

\section{\acl{QCD}}
So far the discussion has covered only the electroweak sector of the \ac{SM} -
leptons, weak gauge bosons and the Higgs particle. The incorporation of
\acl{QCD} at first appears to be quite straightforward. However, it turns out
that performing useful calculations in the theory is a very difficult procedure
- much more so than for the theories presented so far. Here, a brief summary of
the theory will be given with further details available in \cite{pink_book}.

\subsection{Colour}
Inclusion of \ac{QCD} in the \ac{SM} continues in the same manner as for the
electroweak force. The local gauge symmetry group is extended to include
$\SUthree_{C}$ where the $C$ indicates a new conserved quantity - \emph{colour
  charge}. The quarks, of which six flavours are currently known, each carry one
of three colour charges, labelled \emph{red}, \emph{green} and \emph{blue}. As
before, local invariance under this tranformation group requires the
introduction of a covariant derivative and a set of 8 gauge bosons - the gluons.

\subsection{Unique Properties}
The calculational difficulties with \ac{QCD} arise from the fact that the theory
is highly non-linear. This originally stems from the non-Abelian nature of
\SUthree leading to self interaction of the gluons. Whilst this is also true to
some extent in the weak sector, for \ac{QCD} it is considerably more
problematic. Non-linearity makes perturbative calculation within \ac{QCD}
difficult. Whilst other calculation techniques have been developed such as
Lattice \ac{QCD} \cite{lattice_qcd}, it is often the case that \ac{QCD}
processes are poorly understood.

It turns out that \ac{QCD} is most amenable to perturbative calculation at large
energies and short distances. Conversely, the long distance, low energy
behaviour is relatively poorly understood. This is related to two striking
properties of \ac{QCD}.

The first is a phenomenon known as \emph{confinement}. This is a feature
observed by experiment, not derived directly from the theory.  Essentially
confinement forces quarks to exist as bound states - baryons or mesons and
forbids the existence of free quarks.  This is a consequence of the scaling of
the strong force with distance - in particular that the force becomes
arbitrarily strong at large distances, or equivalently, low energies.

A complementary aspect of the scale dependence of the force is known as
\emph{asymptotic freedom}. This causes the strong field to appear very weak at
large energy scales, or equivalently, small distances.
