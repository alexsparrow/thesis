\chapter{Interpretation of Search Results Within Theoretical Models}
\section{Introduction}
It is very often the case that a search for \ac{NP} will yield results
consistent with the currently accepted theory (in most particle physics contexts
this would be the \ac{SM}). In the absence of a \ac{NP} discovery, it is often
desirable to provide additional information in the form of a statistical
interpretation of the results. Such an interpretation should achieve the
following:
\begin{itemize}
\item Indicate the strength of the analysis in searching for the proposed model
  or set of models. This can then be used as an objective measure by which to
  rank different analyses or to benchmark the progress of a single analysis as
  data is collected.
\item Falsify, to some confidence level, a particular theory or some region of
  parameter space within that theory. In the case of a reasonably generic model,
  parameterised in such a way that it may represent other theories (or
  approximate their signature within the detector), theorists may be able to
  test the predictions of a variety of models against the experimental
  results. This will be discussed further in Section~\ref{sec:sms}.
\end{itemize}

Often the choice of how to present a statistical interpretation will require a
trade-off betwen



\section{Statistical Background}
\section{New Physics Models}
\section{Results}