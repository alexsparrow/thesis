\chapter{The CMS Experiment at the Large Hadron Collider}
\section{Introduction}
The Large Hadron Collider (LHC) is a proton-proton ($\Pp\Pp$) accelerator
located at the CERN particle physics laboratory near Geneva, Switzerland. The
LHC is built in the tunnels formerly occupied by the LEP experiment, a
\unit{27}{km} long ring lying on the border between France and Switzerland. Two
beams of protons run in opposite directions around the ring and are made to
collide at four interaction points.

There are four primary experiments at the LHC: \ac{ALICE}, \ac{ATLAS}, \ac{CMS}
and \ac{LHCb}. Each one is constructed around one of the four interaction points
and records the shower of particles produced from the colliding protons. ATLAS
and CMS are large, general purpose detectors designed to search for a variety of
\ac{NP} signatures as well as making higher precision measurements of \ac{SM}
parameters. \ac{ALICE} is designed to examine the products of heavy-ion
collisions (lead-lead) in order to better understand quark gluon plasma and
related physics. The LHCb experiment intends to examine the decays of B mesons
in order to better understand the nature of CP violation.
