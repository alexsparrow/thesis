%% Title
\titlepage[High Energy Physics\\Blackett Laboratory\\Imperial College London]%
{Thesis submitted to Imperial College London\\
  for the degree of Doctor of Philosophy and the Diploma of Imperial College}{2012}

%\input{status.tex}

%% Abstract
\begin{abstract}%[\smaller \thetitle\\ \vspace*{1cm} \smaller {\theauthor}]
  %\thispagestyle{empty}
  This thesis gives an account of two analyses performed using data from the
  \acl{CMS} experiment at the \acl{LHC}. The first analysis measures the
  polarisation of \PW bosons with large tranverse momentum using
  \unit{36}{\invpb} of data collected in 2010. The second applies similar
  techniques to a search for \acl{SUSY} in events containing a single lepton,
  jets and missing transverse energy. This analysis utilises \unit{1.1}{\invfb}
  of data collected up to 2011. Background material related to the \acl{SM},
  \acl{SUSY} and the experimental apparatus are reviewed in detail.

  The \PW polarisation measurement is performed in both the
  $\PW\longrightarrow\Pe\Pneutrino$ and $\PW\longrightarrow\Pmu\Pneutrino$
  channels. The expected effect, a large dominance of the left-handed over the
  right-handed helicity state, is observed with a $7.8\sigma$ significance for
  the \PWp and $5.1\sigma$ for \PWm in the muon channel. Similar results are
  found in the electron channel and for a combined fit to both lepton channels.

  The second analysis conducts a search for \acl{SUSY} in events containing a
  single lepton, jets and missing transverse energy. The search employs
  techniques developed for the \PW polarisation measurement to separate
  supersymmetry from standard model backgrounds. No deviation from the Standard
  Model is observed. A detailed statistical interpretation is performed and used
  to make exclusions within the \acl{CMSSM} as well as two simplified models.
\end{abstract}


%% Declaration
\begin{declaration}
  This dissertation is the result of my own work, except where explicit
  reference is made to the work of others, and has not been submitted
  for another qualification to this or any other university.

  The work presented in \chaps~\ref{sec:wpol}, \ref{sec:susysearch} and
  \ref{sec:interpretation} is the result of collaboration with a number of
  others. For completeness and necessary context, I have presented a full
  account of these analyses. For the \PW polarisation measurement, my
  contributions were primarily to the electron channel and the fitting
  procedure. For the supersymmetry search, I worked mainly on estimating the
  systematic uncertainties and optimising the selection cuts. I also set
  up the statistics procedure detailed in Appendix~\ref{sec:inter_1lepton} and
  performed the interpretation and validation work shown in
  \chap~\ref{sec:interpretation}.

  \vspace*{1cm}
  \begin{flushright}
    Alexander Sparrow
  \end{flushright}
\end{declaration}


%% Acknowledgements
\begin{acknowledgements}
  \acused{CERN}
  I would firstly like to thanks my direct collaborators at \ac{CERN} and
  Imperial College. Firstly, to Georgia Karapostoli for all of her help and
  advice, particularly in helping me to get started on this work. To Markus
  Stoye, Jad Marrouche and Loukas Gouskos for providing so much helpful input
  and fixing so many of my mistakes. Finally, to Oliver Buchmuller and Paris
  Sphicas for their continuous oversight on both analyses and for steering the
  ship through so many storms.

  There are many others deserving of thanks for their continous support and
  advice. First and foremost to my immediate colleagues at \ac{CERN}: Arlo
  Guneratne-Bryer, Michael Cutajar, Zoe Hatherell, Martyn Jarvis, Bryn Mathias,
  Robin Nandi and Nick Wardle for their unending advice, good humour and
  inspiration. Particular thanks to John Jones for the \susyv2 code and many
  interesting discussions. Thanks also to Burt Betchart and Ted Laird for their
  many technical contributions.

  At Imperial College, I'd like to thank Costas Foudas for supervising my
  Masters project and for encouraging me to start a PhD.  I should also like to
  thank my supervisor Alex Tapper. Despite his responsibilities in other areas,
  he has been immsensely supportive and helpful throughouy my PhD.

  The last few years have been among the most enjoyable of my life. This is
  thanks, in no small part, to a number of great friends. Happily, several are
  among the colleagues already mentioned. Many others, in Bath, London and
  Geneva have always been there. For that I am truly thankful. In particular, to
  Paul Schaack for being an excellent friend, housemate and colleague. But most
  of all, to Kitty Liao for her tireless help and support and for putting up
  with me for so long.

  Finally, to my family.  To my two sisters, Maddie and Lydia for the many happy
  times. Above all to my parents, Pauline and Timothy, who taught me the value
  of education. For encouraging me to pursue my dreams and for supporting me
  unwaiveringly at every step along the way.
\end{acknowledgements}


%% Preface
% \begin{preface}
%   This thesis describes my research on various aspects of the \LHCb
%   particle physics program, centred around the \LHCb detector and \LHC
%   accelerator at \CERN in Geneva. Waaay!

%   \noindent
%   % for this example, I'll just mention \ChapterRef{chap:SomeStuff}
%   % and \ChapterRef{chap:MoreStuff}.
% \end{preface}

%% ToC
\tableofcontents

\listoffigures
\listoftables

%% Strictly optional!
% \frontquote%
% {Time is a companion that goes with us on a journey. It reminds us to cherish
%   each moment, because it will never come again. What we leave behind is not as
%   important as how we have lived.}%
% {Jean-Luc Picard, 2371, after the destruction of the Enterprise-D}
