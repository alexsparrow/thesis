\documentclass{mythesis}
\usepackage{mythesis}

%% You can set the line spacing this way
%\setallspacing{double}
%% or a section at a time like this
%\setfrontmatterspacing{double}

%% PDF metadata
\makeatletter
\@ifpackageloaded{hyperref}{%
\hypersetup{%
pdftitle = {Studying B to K pi decays with LHCb},
pdfsubject = {Andy Buckley's PhD thesis},
pdfkeywords = {LHCb, B, physics, LHC, heavy flavour},
pdfauthor = {\textcopyright\ Andy Buckley}
}
}{}
\makeatother

%% Define the thesis title and author
\title{A study of \BToKPi decays with\\ the \LHCb experiment}
\author{Andrew Gordon Buckley}

%% Start the document
\begin{document}

%% Define the un-numbered front matter (cover pages, rubrik and table of contents)
\begin{frontmatter}
  %% Title
\titlepage[of Imperial College London]%
{A dissertation submitted to Imperial College London\\
 for the degree of Doctor of Philosophy}

\input{status.tex}

%% Abstract
\begin{abstract}%[\smaller \thetitle\\ \vspace*{1cm} \smaller {\theauthor}]
  %\thispagestyle{empty}
  Abstract to go here
\end{abstract}


%% Declaration
\begin{declaration}
  This dissertation is the result of my own work, except where explicit
  reference is made to the work of others, and has not been submitted
  for another qualification to this or any other university.

  The work presented in Chapters~\ref{sec:wpol}, \ref{sec:susysearch},
  \ref{sec:interpretation} is the result of collaboration with a number of
  others. For completeness and necessary context, I have presented a full
  account of these analyses. For the \PW polarisation measurement, my
  contributions were primarily to the electron channel measurement and the
  fitting procedure. For the supersymmetry search, I worked mostly on the
  estimation of the systematic uncertainties and optimising some of the
  selection cuts. I also set up the statistics procedure detailed in
  Appendix~\ref{sec:inter_1lepton} and performed the interpretation and
  validation work shown in Chapter~\ref{sec:interpretation}.

  \vspace*{1cm}
  \begin{flushright}
    Alexander Sparrow
  \end{flushright}
\end{declaration}


%% Acknowledgements
\begin{acknowledgements}
  Of the many people who deserve thanks, some are particularly prominent,
  such as my supervisor\dots
\end{acknowledgements}


%% Preface
\begin{preface}
  This thesis describes my research on various aspects of the \LHCb
  particle physics program, centred around the \LHCb detector and \LHC
  accelerator at \CERN in Geneva. Waaay!

  \noindent
  % for this example, I'll just mention \ChapterRef{chap:SomeStuff}
  % and \ChapterRef{chap:MoreStuff}.
\end{preface}

%% ToC
\tableofcontents

\listoffigures
\listoftables

%% Strictly optional!
\frontquote%
{Time is a companion that goes with us on a journey. It reminds us to cherish
  each moment, because it will never come again. What we leave behind is not as
  important as how we have lived.}%
{Jean-Luc Picard, 2371, after the destruction of the Enterprise-D}

\end{frontmatter}
2TEST
%% Start the content body of the thesis
\begin{mainmatter}
  %% Actually, more semantic chapter filenames are better, like "chap-bgtheory.tex"
  \input{ch1-theory}
  \input{ch2-expt}
  \input{ch3-wpol}
  \input{ch4-susy}
  \input{ch5-interpretation}

  %% To ignore a specific chapter while working on another,
  %% making the build faster, comment it out like this:
  %\input{chap4}
\end{mainmatter}

%% Produce the appendices
\begin{appendices}
  \input{appendices}
\end{appendices}

%% Produce the un-numbered back matter (e.g. colophon,
%% bibliography, tables of figures etc., index...)
\begin{backmatter}
  
\bibliographystyle{lucas_unsrt}
\bibliography{thesis.bib}
\end{backmatter}

%% Close
\end{document}
