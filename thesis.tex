\documentclass{mythesis}
\usepackage{mythesis}

%% You can set the line spacing this way
%\setallspacing{double}
%% or a section at a time like this
%\setfrontmatterspacing{double}

%% PDF metadata
\makeatletter
\@ifpackageloaded{hyperref}{%
\hypersetup{%
pdftitle = {Studying B to K pi decays with LHCb},
pdfsubject = {Andy Buckley's PhD thesis},
pdfkeywords = {LHCb, B, physics, LHC, heavy flavour},
pdfauthor = {\textcopyright\ Andy Buckley}
}
}{}
\makeatother

%% Define the thesis title and author
\title{Measuring the Polarisation of the \PW boson and Searching for Supersymmetry at CMS}
\author{Alexander Sparrow}

\graphicspath{generated//}

%% Start the document
\begin{document}

%% Define the un-numbered front matter (cover pages, rubrik and table of contents)
\begin{frontmatter}
  %% Title
\titlepage[of Imperial College London]%
{A dissertation submitted to Imperial College London\\
 for the degree of Doctor of Philosophy}

\input{status.tex}

%% Abstract
\begin{abstract}%[\smaller \thetitle\\ \vspace*{1cm} \smaller {\theauthor}]
  %\thispagestyle{empty}
  Abstract to go here
\end{abstract}


%% Declaration
\begin{declaration}
  This dissertation is the result of my own work, except where explicit
  reference is made to the work of others, and has not been submitted
  for another qualification to this or any other university. This
  dissertation does not exceed the word limit for the respective Degree
  Committee.
  \vspace*{1cm}
  \begin{flushright}
    Andy Buckley
  \end{flushright}
\end{declaration}


%% Acknowledgements
\begin{acknowledgements}
  Of the many people who deserve thanks, some are particularly prominent,
  such as my supervisor\dots
\end{acknowledgements}


%% Preface
\begin{preface}
  This thesis describes my research on various aspects of the \LHCb
  particle physics program, centred around the \LHCb detector and \LHC
  accelerator at \CERN in Geneva. Waaay!

  \noindent
  % for this example, I'll just mention \ChapterRef{chap:SomeStuff}
  % and \ChapterRef{chap:MoreStuff}.
\end{preface}

%% ToC
\tableofcontents

%% Strictly optional!
\frontquote%
  {Time is a companion that goes with us on a journey. It reminds us to cherish each moment, because it will never come again. What we leave behind is not as important as how we have lived.}%
  {Jean-Luc Picard, 2371, after the destruction of the Enterprise-D}

\end{frontmatter}
%% Start the content body of the thesis
\begin{mainmatter}
  %% Actually, more semantic chapter filenames are better, like "chap-bgtheory.tex"
  \subimport{1-theory/}{theory}
  \input{ch2-expt}
  \input{ch3-wpol}
  \input{ch4-susy}
  \input{ch5-interpretation}

  %% To ignore a specific chapter while working on another,
  %% making the build faster, comment it out like this:
  %\input{chap4}
\end{mainmatter}

%% Produce the appendices
\begin{appendices}
  
\chapter{Kinematics}
A Lorentz tranformation can be written as
\begin{equation}
\left(\begin{array}{c} E' \\ p_{\parallel}' \end{array} \right)
=
\left(
\begin{array}{cc}
\gamma & -\gamma\beta \\
-\gamma\beta & \gamma
\end{array}
\right)
\left (\begin{array}{c} E \\ p_{\parallel} \end{array}\right)
\end{equation}
and
\begin{equation}
p_{\perp}' = p_{\perp}
\end{equation}

Boosting from a particles rest frame into the lab frame,
\begin{equation}
\left(\begin{array}{c} E \\ P \end{array} \right)
=
\left(
\begin{array}{cc}
\gamma & -\gamma\beta \\
-\gamma\beta & \gamma
\end{array}
\right)
\left (\begin{array}{c} M \\ 0 \end{array}\right)
\end{equation}
and so
\begin{eqnarray*}
E &=& \gamma M  \Longrightarrow \gamma = \frac{E}{M} \\
|P| &=& \gamma\beta M \Longrightarrow \beta = \frac{|P|}{\gamma M} = \frac{|P|}{E}
\end{eqnarray*}
also
\begin{eqnarray*}
\gamma &=& \frac{\sqrt{P + M}}{M} \\
&=& \sqrt{1 +\left(\frac{|P|}{M}\right)^2}
\end{eqnarray*}

\end{appendices}

%% Produce the un-numbered back matter (e.g. colophon,
%% bibliography, tables of figures etc., index...)
\begin{backmatter}
  \bibliographystyle{lucas_unsrt}
\bibliography{shorttitles,bibstrings,thesis}

\begin{colophon}
  A number of software tools were vital to the production of this thesis. The
  author is extremely grateful to the many individiuals responsible. Without
  such high-quality tools, this work would simply not have been possible.

The thesis was written using the \textsc{Emacs} text-editor and typeset with
\TeX/\LaTeX using the \textsc{HEPThesis} package. The \texttt{git} version
control system was invaluable for maintaining backups and managing changes.

All work was performed using the \textsc{GNU/Linux} operating system, primarily
the variant assembled by the volunteers of the \textsc{Arch Linux} project. Much
of the code was written using the \textsc{Python} programming language. Whilst
some plots were produced using the \textsc{ROOT} framework, the author
recommends \texttt{matplotlib} for its superior API and powerful features.
\end{colophon}

\end{backmatter}

%% Close
\end{document}
