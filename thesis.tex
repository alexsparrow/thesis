\documentclass{mythesis}
\usepackage{mythesis}

%% You can set the line spacing this way
%\setallspacing{double}
%% or a section at a time like this
%\setfrontmatterspacing{double}

%% PDF metadata
\makeatletter
\@ifpackageloaded{hyperref}{%
\hypersetup{%
pdftitle = {Studying B to K pi decays with LHCb},
pdfsubject = {Andy Buckley's PhD thesis},
pdfkeywords = {LHCb, B, physics, LHC, heavy flavour},
pdfauthor = {\textcopyright\ Andy Buckley}
}
}{}
\makeatother

%% Define the thesis title and author
\title{Measuring the Polarisation of the \PW boson and Searching for
  Supersymmetry at CMS} \author{Alexander Sparrow}

%% Start the document
\begin{document}

\begin{acronym}
\acro{NP}{New Physics}
\acro{SM}{Standard Model}
\acro{ALICE}{A Large Ion Collider Experiment}
\acro{ATLAS}{A Toroidal LHC Apparatus}
\acro{LHCb}{Large Hadron Collider Beauty Experiment}
\acro{CMS}{Compact Muon Solenoid}
\acro{LHC}{Large Hadron Collider}
\end{acronym}

%% Define the un-numbered front matter (cover pages, rubrik and table of contents)
\begin{frontmatter}
  %% Title
\titlepage[of Imperial College London]%
{A dissertation submitted to Imperial College London\\
 for the degree of Doctor of Philosophy}

\input{status.tex}

%% Abstract
\begin{abstract}%[\smaller \thetitle\\ \vspace*{1cm} \smaller {\theauthor}]
  %\thispagestyle{empty}
  Abstract to go here
\end{abstract}


%% Declaration
\begin{declaration}
  This dissertation is the result of my own work, except where explicit
  reference is made to the work of others, and has not been submitted
  for another qualification to this or any other university.

  The work presented in Chapters~\ref{sec:wpol}, \ref{sec:susysearch},
  \ref{sec:interpretation} is the result of collaboration with a number of
  others. For completeness and necessary context, I have presented a full
  account of these analyses. For the \PW polarisation measurement, my
  contributions were primarily to the electron channel measurement and the
  fitting procedure. For the supersymmetry search, I worked mostly on the
  estimation of the systematic uncertainties and optimising some of the
  selection cuts. I also set up the statistics procedure detailed in
  Appendix~\ref{sec:inter_1lepton} and performed the interpretation and
  validation work shown in Chapter~\ref{sec:interpretation}.

  \vspace*{1cm}
  \begin{flushright}
    Alexander Sparrow
  \end{flushright}
\end{declaration}


%% Acknowledgements
\begin{acknowledgements}
  Of the many people who deserve thanks, some are particularly prominent,
  such as my supervisor\dots
\end{acknowledgements}


%% Preface
\begin{preface}
  This thesis describes my research on various aspects of the \LHCb
  particle physics program, centred around the \LHCb detector and \LHC
  accelerator at \CERN in Geneva. Waaay!

  \noindent
  % for this example, I'll just mention \ChapterRef{chap:SomeStuff}
  % and \ChapterRef{chap:MoreStuff}.
\end{preface}

%% ToC
\tableofcontents

\listoffigures
\listoftables

%% Strictly optional!
\frontquote%
{Time is a companion that goes with us on a journey. It reminds us to cherish
  each moment, because it will never come again. What we leave behind is not as
  important as how we have lived.}%
{Jean-Luc Picard, 2371, after the destruction of the Enterprise-D}

\end{frontmatter}
%% Start the content body of the thesis
\begin{mainmatter}
  \linenumbers
  %% Actually, more semantic chapter filenames are better, like "chap-bgtheory.tex"
  \import{1-theory/}{theory}
  \import{2-expt/}{expt}
  \import{3-wpol/}{wpol}
  \import{4-susy/}{susy}
  \import{5-interpretation/}{interpretation}

  %% To ignore a specific chapter while working on another,
  %% making the build faster, comment it out like this:
  %\input{chap4}
  \nolinenumbers
\end{mainmatter}

%% AS: Added to make scons seen these files (it doesn't parse the import
%% statements above!)
\begin{comment}
  
\chapter{The Standard Model}
\section{Introduction}
It has been written before that the \acsl{SM} of particle physics is perhaps the
most thoroughly tested theory of nature ever constructed. Despite decades of experimental effort,

  \chapter{The CMS Experiment at the Large Hadron Collider}
\section{Introduction}
The Large Hadron Collider (LHC) is a proton-proton ($\Pp\Pp$) accelerator
located at the CERN particle physics laboratory near Geneva, Switzerland. The
LHC is built in the tunnels formerly occupied by the LEP experiment, a
\unit{27}{km} long ring lying on the border between France and Switzerland. Two
beams of protons run in opposite directions around the ring and are made to
collide at four interaction points.

There are four primary experiments at the LHC: \ac{ALICE}, \ac{ATLAS}, \ac{CMS}
and \ac{LHCb}. Each one is constructed around one of the four interaction points
and records the shower of particles produced from the colliding protons. ATLAS
and CMS are large, general purpose detectors designed to search for a variety of
\ac{NP} signatures as well as making higher precision measurements of \ac{SM}
parameters. \ac{ALICE} is designed to examine the products of heavy-ion
collisions (lead-lead) in order to explore the quark gluon plasma and related
physics. Finally, the \ac{LHCb} experiment is optimised for the study of B-meson
decays. These are important for the study of CP violation within the \ac{SM} but
might also provide potential avenues for the discovery of \ac{NP}.

\section{The \acl{LHC}}

\section{The \acl{CMS} Experiment}



  \chapter{Measuring the Helicity Polarisation of the $\PW$ Boson}
\section{Introduction}
The study of \Wjets production at a hadron collider presents an important
opportunity for furthering understanding of the underlying Electroweak and
\ac{QCD} processes. In particular, since it is one of a relatively small number
of processes for which highly precise \ac{NLO} calculations have been performed,
experimental measurements can give a direct constraint on the \acp{PDF}. \Wjets
production is also of considerable interest in the context of \ac{NP} searches
where these events are often a dominant background. Finally, the neutrino in the
leptonic decay mode provides a source of ``real'' missing energy which can be
useful in the understanding of detector effects relevant to searches for
\acs{WIMP}-type particles present in \ac{SUSY} and other theories.

\section{Background}
Some theoretical background relating to \PW helicity effects has been presented
in Section~TODO. Here, the discussion will be oriented towards a more
experimental context.

For small values of \PW transverse momentum, \PtW the differential angular
cross-section for the process
$\Pp\Pp\longrightarrow\PWpm\longrightarrow\Plpm\Pgnl$ follows the Drell-Yan
distribution
\begin{equation}
\frac{dN}{d(\cos\theta)} \propto (1\mp \cos\theta)^2
\end{equation}

It is well known from straightforward helicity arguments\cite{mirkes_w_1994}that
\PW produced along the beam axis will exhibit a 100\% left-handed polarization. This
can be seen by considering the leading order partonic subprocesses
\begin{equation}
\Pup\APdown \longrightarrow \PWp \qquad\textrm{and}\qquad
\Pdown\APup\longrightarrow\PWm
\end{equation}
Firstly, note that the fraction of the proton momentum carried by the quark (as
determined by the \aclp{PDF}) is greater than that of the anti-quark. In
addition given that the \ac{LHC} is a $\Pp\Pp$ collider, valence anti-quarks are
not present. Anti-quarks must be drawn from the sea and are therefore likely to
be low momentum. Taking these two facts together, the quark is very likely to be
higher momentum than the anti-quark. By momentum conservation, it is expected
that the \PW will be produced overwhelmingly in the direction of the original
quark. Then given the \VminusA nature of the weak interaction, it is seen that
the quark must be left-handed and, by helicity conservation the \PW will be
polarised nearly 100\% left-handedly along the beam axis. A small dilution will
occur in instances where the anti-quark has by chance a larger momentum fraction
than the quark.

It is worth mentioning that the situation is not identical at the Tevatron
$\Pp\APp$ collider. Although the \PWp also possess a 100\% left-handed
polarisation along the beamline (via similar arguments to those given above),
the \PWm are found to have a near 100\% right-handed polarisation. This is a
result of the subprocess $\APup\Pdown\longrightarrow\PWm$ where this time the
\APup carries more momentum.

In the case, where the \PW carries a significant transverse momentum \PtW, the
situation is more complex.


\begin{figure}
\centering
\subfloat[]{\includegraphics[width=0.3\textwidth]{fig/wpol_prod_a}}\quad
\subfloat[]{\includegraphics[width=0.3\textwidth]{fig/wpol_prod_b}}\quad
\subfloat[]{\includegraphics[width=0.3\textwidth]{fig/wpol_prod_c}}\\
\subfloat[]{\includegraphics[width=0.3\textwidth]{fig/wpol_prod_d}}\quad
\subfloat[]{\includegraphics[width=0.3\textwidth]{fig/wpol_prod_e}}\quad
\subfloat[]{\includegraphics[width=0.3\textwidth]{fig/wpol_prod_f}}
\caption{Illustrations of $\PWplus+1$~jet production modes at the LHC. The
  $\Pgluon$ superscript indicates its helicity}
\label{fig:w1jet_modes}
\end{figure}
  \chapter{\acl{SUSY}}
\label{sec:susy}
The \ac{SM}, as described in Chapter~\ref{sec:sm} appears to describe all the
known fundamental particles and interactions to an incredible degree of
accuracy. What cause is there to believe that there might be physical phenomena
not described by this theory? This will be the topic we now turn to.

\section{Beyond the \acl{SM}}
A limitation immediately apparent in the \ac{SM} is that it makes no attempt to
unify gravity with the other fundamental forces. From a purely experimental
perspective, this is not an issues, since no experiment is able to explore
gravitational effects at the quantum scale. This is not likely to change in the
forseeable future. However, it seems certain to many theorists that a quantised
theory of gravity must exist and indeed this has been the focus of great
theoretical effort in the last thirty years. Several potential theories have
emerged, aiming to provide an entirely unified picture of fundamental physics;
two examples being \emph{string theory} and \emph{loop quantum gravity}. Whilst
proponents of these theories have been criticised for devising untestable
hypotheses, it ``feel right'' to many physisicists that new physics must be
present to give a more unified physical theory.

\subsection{The Hierarchy Problem}
The \emph{hierarchy problem} is arguably one of the strongest theoretical
arguments for physics beyond the \ac{SM}. This relates to the apparently huge
difference between the weak mass scale (\Mweak) and the Planck scale of gravity
(\Mplanck) - over 16 orders of magnitude. To some, it seems unthinkable that no
new physics should appear in this vast range of energies.

As well as being aesthetically undesirable, the hierarchy problem presents a
real theoretical issue for the mass of the Higgs boson. The Higgs boson mass
receives quantum corrections from every particle that it couples to - directly
or indirectly. These corrections have the form,
\begin{equation}
\Delta \mHiggs^2 = -\frac{\left|\lambda_f\right|^2}{8\pi^2}\LambdaUV^2 + \ldots
\end{equation}
where $\lambda_f$ is a coupling constant to a fermion $f$ and \LambdaUV is the
momentum cut-off regulating the loop integral. All \ac{SM} fermions can
contribute to this correction, which is largest for the top quark with
$\lambda_f \approx 1$. Interpreting $\LambdaUV$ as the scale at which new
physics should appear to alter the behaviour of the theory and taking this to be
the Planck scale, these corrections are found to be 30 orders of magnitude
larger than the expected Higgs mass ($\approx \unit{100}{\GeV}$).

Whilst it might seem possible to just pick a small value of \LambdaUV, this
would require some form of new physics at this scale to alter the propagators in
the loop as well as cutting off the loop integral. As will be seen \ac{SUSY}
provides a neat solution to this problem.

\subsection{Dark Matter}
The problem of Dark Matter is perhaps the most convincing argument, at least to
experimentalists for the existence of some physics beyond the \ac{SM}. It was
observed as early as 1932 \cite{darkmatter_review} that galactic rotation curves
appeared to be at odds with those predicted from an estimation of their visible
mass. This seems to suggest a great deal of additional mass is present in the
galaxy, over and above that which can be inferred from the visible matter. This
observation is confirmed by measurements of gravitational lensing
\cite{bullet_cluster} and mapping of the cosmic microwave background
\cite{wmap_7year}. Current observations suggest dark matter comprises more than
80\% (TODO:citation needed) of the matter content of the universe. No
experimentally confirmed theory is able to match such a prediction.

Because of its invisible nature, a possible explanation for Dark Matter is a
\acl{WIMP}. Experiment hoping to directly detect such a particle have been
underway for some time. Typically, a large volume of a suitable gas or liquid is
used in the hope that passing \acp{WIMP} will undergo a nuclear
interaction. Whilst discoveries have been claimed, the evidence is not yet
believe to be conclusive \cite{dama_libra}.

A related issue is that of Dark Energy - believed to constitute nearly
three-quarters of the mass-energy content of the universe. This is an effect not
predicted by the currently accepted theories of particle physics. Taken
together, these phenomena are strong evidence of physics beyond the \ac{SM}.

\section{An Additional Symmetry}
\section{\ac{SUSY} Particles}
\section{\ac{SUSY} Phenomenology}

  \chapter{Interpretation of Search Results Within Theoretical Models}
\end{comment}

%% Produce the appendices
\begin{appendices}
  \input{appendices}
\end{appendices}

%% Produce the un-numbered back matter (e.g. colophon,
%% bibliography, tables of figures etc., index...)
\begin{backmatter}
  
\bibliographystyle{lucas_unsrt}
\bibliography{thesis.bib}
\end{backmatter}

%% Close
\end{document}
