\chapter{\acl{SUSY}}
\label{sec:susy}
The \ac{SM}, as described in Chapter~\ref{sec:sm} appears to describe all the
known fundamental particles and interactions to an incredible degree of
accuracy. What cause is there to believe that there might be physical phenomena
not described by this theory? This will be the topic we now turn to.

\section{Beyond the \acl{SM}}
A limitation immediately apparent in the \ac{SM} is that it makes no attempt to
unify gravity with the other fundamental forces. From a purely experimental
perspective, this is not an issues, since no experiment is able to explore
gravitational effects at the quantum scale. This is not likely to change in the
forseeable future. However, it seems certain to many theorists that a quantised
theory of gravity must exist and indeed this has been the focus of great
theoretical effort in the last thirty years. Several potential theories have
emerged, aiming to provide an entirely unified picture of fundamental physics;
two examples being \emph{string theory} and \emph{loop quantum gravity}. Whilst
proponents of these theories have been criticised for devising untestable
hypotheses, it ``feel right'' to many physisicists that new physics must be
present to give a more unified physical theory.

\subsection{The Hierarchy Problem}
The \emph{hierarchy problem} is arguably one of the strongest theoretical
arguments for physics beyond the \ac{SM}. This relates to the apparently huge
difference between the weak mass scale (\Mweak) and the Planck scale of gravity
(\Mplanck) - over 16 orders of magnitude. To some, it seems unthinkable that no
new physics should appear in this vast range of energies.

As well as being aesthetically undesirable, the hierarchy problem presents a
real theoretical issue for the mass of the Higgs boson. The Higgs boson mass
receives quantum corrections from every particle that it couples to - directly
or indirectly. These corrections have the form,
\begin{equation}
\Delta \mHiggs^2 = -\frac{\left|\lambda_f\right|^2}{8\pi^2}\LambdaUV^2 + \ldots
\end{equation}
where $\lambda_f$ is a coupling constant to a fermion $f$ and \LambdaUV is the
momentum cut-off regulating the loop integral. All \ac{SM} fermions can
contribute to this correction, which is largest for the top quark with
$\lambda_f \approx 1$. Interpreting $\LambdaUV$ as the scale at which new
physics should appear to alter the behaviour of the theory and taking this to be
the Planck scale, these corrections are found to be 30 orders of magnitude
larger than the expected Higgs mass ($\approx \unit{100}{\GeV}$).

Whilst it might seem possible to just pick a small value of \LambdaUV, this
would require some form of new physics at this scale to alter the propagators in
the loop as well as cutting off the loop integral. As will be seen \ac{SUSY}
provides a neat solution to this problem.

\subsection{Dark Matter}
The problem of Dark Matter is perhaps the most convincing argument, at least to
experimentalists for the existence of some physics beyond the \ac{SM}. It was
observed as early as 1932 \cite{darkmatter_review} that galactic rotation curves
appeared to be at odds with those predicted from an estimation of their visible
mass. This seems to suggest a great deal of additional mass is present in the
galaxy, over and above that which can be inferred from the visible matter. This
observation is confirmed by measurements of gravitational lensing
\cite{bullet_cluster} and mapping of the cosmic microwave background
\cite{wmap_7year}. Current observations suggest dark matter comprises more than
80\% (TODO:citation needed) of the matter content of the universe. No
experimentally confirmed theory is able to match such a prediction.

Because of its invisible nature, a possible explanation for Dark Matter is a
\acl{WIMP}. Experiment hoping to directly detect such a particle have been
underway for some time. Typically, a large volume of a suitable gas or liquid is
used in the hope that passing \acp{WIMP} will undergo a nuclear
interaction. Whilst discoveries have been claimed, the evidence is not yet
believe to be conclusive \cite{dama_libra}.

A related issue is that of Dark Energy - believed to constitute nearly
three-quarters of the mass-energy content of the universe. This is an effect not
predicted by the currently accepted theories of particle physics. Taken
together, these phenomena are strong evidence of physics beyond the \ac{SM}.

\section{An Additional Symmetry}
\section{\ac{SUSY} Particles}
\section{\ac{SUSY} Phenomenology}
