\chapter*{Conclusion}
\addcontentsline{toc}{chapter}{Conclusion}

As described in the introduction, the overall theme of this work has been the
search for some sign of physics beyond the \ac{SM}. More specifically, a search
for \ac{SUSY} or \ac{SUSY}-like theories which predict the existence of a heavy,
stable, weakly interacting particle - the \ac{LSP}. As was seen, these theories
are attractive for a number of reasons, not least that they provide an answer to
the Dark Matter problem described in Chapter~\ref{sec:susy}.

The search for new physics was heavily inspired by a previous measurement
undertaken with data taken at \ac{CMS} during the 2010 run. This sought to
measure the polarisation of \PW bosons with large transverse momentum at the
\ac{LHC}. An enriched sample of $\PW\longrightarrow\Plepton\Pneutrino$ events
with large transverse momentum was selected from \unit{36}{\invpb} of
data.

In order to measure the polarisation, we wish to measure
\costhetastar. Unfortunately, the unknown boost of the initial partons at a
hadron collider renders this impossible. Instead, a variable found to be highly
correlated with \costhetastar is used - \LP. Shape templates are constructed
from simulated $\PW\longrightarrow\Plepton\Pneutrino$ events. Each gives the
shape of \LP for left-handed, right-handed and longitudinally polarised \PW
bosons. With the inclusion of appropriate templates for the remaining background
processes, the helicity fractions are extracted via a binned maximum likelihood
fit. The results are expressed as the difference between the left-handed and
right-handed helicity fractions, \fLmfR and the longitudinal polarisation \fo.

Independent measurements are performed for each lepton charge and
flavour. Finally, a combined measurement is performed using both lepton channels
simultaneously. These confirm the existence of a novel effect at the \ac{LHC} -
namely that the left-handed polarisation states come to dominate over the
right-handed at large \PW transverse momentum. Furthermore the values of \fLmfR
and \f0 appear to agree within errors with theoretical predictions.

In addition to being a useful and novel confirmation of the \ac{SM}, the \PW
polarisation measurement provided a powerful set of technniques for undertaking
a \ac{SUSY} search in events containing a missing transverse energy, jets and a
single lepton. The addition of a lepton serves to suppress the \ac{SM}
backgrounds at the cost of some sensitivity.

The \LP variable from the \PW polarisation analysis is used to discrimate
\ac{SM} and \ac{SUSY} events. Furthermore, an additional variable is used to
parameterise the scale of the interaction - \STlep. A search is performed and no
excess over the expected \ac{SM} backgrounds is observed.