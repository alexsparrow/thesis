\addcontentsline{toc}{chapter}{Conclusion}
\chapter*{Conclusion}
\markboth{Conclusion}{Conclusion}
As described in the introduction, the overall theme of this work has been the
search for some sign of physics beyond the \ac{SM}. More specifically, a search
for \ac{SUSY} or \ac{SUSY}-like theories which predict the existence of a heavy,
stable \ac{WIMP}. As was seen, these theories are attractive for a number of
reasons, not least that they provide an answer to the Dark Matter problem
described in \chap~\ref{sec:susy}.

The search for new physics was inspired by a previous measurement undertaken
using data taken at \ac{CMS} during 2010. This sought to measure the
polarisation of \PW bosons with large transverse momentum at the \ac{LHC}. An
enriched sample of $\PW\longrightarrow\Plepton\Pneutrino$ events with large
transverse momentum was selected from \unit{36}{\invpb} of data.

The \LP variable was devised to probe the $\cos\thetastar$ distribution of the
\PW boson. Shape templates in this variable were constructed from simulated
$\PW\longrightarrow\Plepton\Pneutrino$ events. Each gives the shape of \LP for
left-handed, right-handed and longitudinally polarised \PW bosons. With the
inclusion of appropriate templates for the remaining background processes, the
helicity fractions have been extracted via a binned maximum likelihood fit. The
results are expressed as the difference between the left-handed and right-handed
helicity fractions, \fLmfR, and the longitudinal polarisation, \f0.

Independent measurements were performed for each lepton charge and flavour. A
combined measurement has also been performed using both lepton channels. These
confirm the existence of a novel effect at the \ac{LHC} - namely that the
left-handed polarisation states come to dominate over the right-handed at large
\PtW. Furthermore the values of \fLmfR and \f0 appear to agree, within
uncertainties, with theoretical predictions~\cite{berger_left_handed_w}.

In addition to being a useful and novel confirmation of the \ac{SM}, the \PW
polarisation measurement provides a powerful set of techniques for undertaking a
\ac{SUSY} search in events containing a missing transverse energy, jets and a
single lepton.

The \LP variable from the \PW polarisation analysis is used to discriminate
\ac{SUSY} events from \ac{SM} backgrounds. An additional variable, \STlep is
used to parameterise the scale of the interaction. A search has been performed
with \unit{1.1}{\invfb} of data at \ac{CMS}. No excess over expected \ac{SM}
backgrounds has been observed.

Finally, this null observation has been used to set limits in a number of
\ac{NP} models. For this, considerable effort was invested in constructing a
suitable likelihood model capturing all statistical and systematic
effects. Within the context of the \ac{CMSSM} -- a standard benchmark for
\ac{SUSY} searches -- squark masses below $\approx \unit{900}{\GeV}$ and gluino
masses below $\approx \unit{500}{\GeV}$ have been excluded at 95\% confidence.

In addition to the \ac{CMSSM} exclusion, two simplified models were selected -
\TthreeW and \Ttwott. The \TthreeW model considers events arising from
pair-production of a gluino type particle which then decay to the \ac{LSP} via
an intermediate particle. Limits in the \TthreeW model exclude the parameter
space $\Mgluino < \unit{600}{\GeV}$, $\Mlsp < \unit{200}{\GeV}$ when the mass of
the intermediate particle is close to that of the \ac{LSP}. This assumes a
gluino production cross-section with \ac{QCD} strength couplings. These limits
are seen to vary only slightly with the mass of the intermediate particle.

The \Ttwott model considers events initiated by pair production of stop squarks
(or similar), both decaying directly to the \ac{LSP}. This is inspired by
theoretically appealing \ac{SUSY} models in which the stop is light. Whilst an
upper limit on the cross-section is set, no exclusion is possible with respect
to that predicted by \ac{QCD}.

In summary, a precision measurement of the \ac{SM} has been performed
in addition to a search for new physics. Whilst new light has been shed on
well-known physics, no statistically significant deviation or excess has been
observed.
