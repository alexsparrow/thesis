\chapter*{Conclusion}
\addcontentsline{toc}{chapter}{Conclusion}

As described in the introduction, the overall theme of this work has been the
search for some sign of physics beyond the \ac{SM}. More specifically, a search
for \ac{SUSY} or \ac{SUSY}-like theories which predict the existence of a heavy,
stable, weakly interacting particle - the \ac{LSP}. As was seen, these theories
are attractive for a number of reasons, not least that they provide an answer to
the Dark Matter problem described in Chapter~\ref{sec:susy}.

The search for new physics was heavily inspired by a previous measurement
undertaken with data taken at \ac{CMS} during the 2010 run. This sought to
measure the polarisation of \PW bosons with large transverse momentum at the
\ac{LHC}. An enriched sample of $\PW\longrightarrow\Plepton\Pneutrino$ events
with large transverse momentum was selected from \unit{36}{\invpb} of
data.

In order to measure the polarisation, we wish to measure
$\cos\thetastar$. Unfortunately, the unknown boost of the initial partons at a
hadron collider renders this impossible. Instead, a variable found to be highly
correlated with $\cos\thetastar$ is used - \LP. Shape templates are constructed
from simulated $\PW\longrightarrow\Plepton\Pneutrino$ events. Each gives the
shape of \LP for left-handed, right-handed and longitudinally polarised \PW
bosons. With the inclusion of appropriate templates for the remaining background
processes, the helicity fractions are extracted via a binned maximum likelihood
fit. The results are expressed as the difference between the left-handed and
right-handed helicity fractions, \fLmfR and the longitudinal polarisation \fo.

Independent measurements are performed for each lepton charge and
flavour. Finally, a combined measurement is performed using both lepton channels
simultaneously. These confirm the existence of a novel effect at the \ac{LHC} -
namely that the left-handed polarisation states come to dominate over the
right-handed at large \PW transverse momentum. Furthermore the values of \fLmfR
and \f0 appear to agree within errors with theoretical predictions.

In addition to being a useful and novel confirmation of the \ac{SM}, the \PW
polarisation measurement provided a powerful set of technniques for undertaking
a \ac{SUSY} search in events containing a missing transverse energy, jets and a
single lepton. The addition of a lepton serves to suppress the \ac{SM}
backgrounds at the cost of some sensitivity.

The \LP variable from the \PW polarisation analysis is used to discrimate
\ac{SM} and \ac{SUSY} events. Furthermore, an additional variable is used to
parameterise the scale of the interaction - \STlep. A search has been performed
with \unit{1.1}{\infb} of data at \ac{CMS}. No excess over the expected \ac{SM}
backgrounds is observed.

Finally, the null observation has been used to set limits in a number of new
physics models. For this, considerable effort was invested in constructing a
suitable likelihood model capturing all of the systematic effects. Within the
context of the \ac{CMSSM} - a standard benchmark for \ac{SUSY} searches - squark
masses below $\approx \unit{900}{\GeV}$ and gluino masses below $\approx
\unit{500}{\GeV}$ are excluded at 95\% confidence.

In addition to the \ac{CMSSM} exclusion, two simplified models were selected -
\TthreeW and \Ttwott. The \TthreeW model considers events arising from
pair-production of a gluino type particle decaying to the \ac{LSP} via an
intermediate particle. Limits in the \TthreeW model exclude the parameter space
$\Mgluino < \unit{600}{\GeV}$, $\Mlsp < \unit{200}{\GeV}$ assuming the gluino
production cross-section due to \ac{QCD} strength couplings. These limits vary
slightly with the mass of the intermediate particle.

The \Ttwott model considers events iniated by pair production of stop squarks
(or similar), both decaying directly to the \ac{LSP}. This is inspired by
\ac{SUSY} models containing a light stop. Whilst an upper limit on the
cross-section is set, no exclusion is possible with respect to the cross-section
predicted by \ac{QCD}.

To finally conclude, a precision measurement of the \ac{SM} has been performed
in addition to a search for new physics. Whilst new light has been shed on
well-known physics, no statistically significant deviation or excess has been
observed.
