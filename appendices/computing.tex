\chapter{Computing}
\section{SUSYv2}
My largest contribution to computing was in the maintenance and further
development of the \ac{SUSYv2} software package. This is a standalone
\ac{ROOT}-based analysis framework written in a mixture of C++ and Python. It
has been used in a number of analyses by the Imperial College and Friends group
within \ac{CMS}. These have primarily been \ac{SUSY}-based analyses, searching
in the 0, 1 and 2 lepton final states, photon final states. It has also been
used for the \PW polarisation measurement as well as several other projects. The
vast majority of the initial code was written by Dr. John Jones with subsequent
contributions from a number of others.

The \ac{SUSYv2} code aims to minimise the number of reads performed on a
\ac{ROOT} tree by reading branches on demand and performing lazy calculations as
required to satisfy analysis code requests for higher-level observables. The
dependency chain between calculated quantities and ultimately \ac{ROOT}-tree
branches is encoded via function calls and all intermediate values are memoized
or cached for a single event to avoid recalculation or rereading of an ntuple
branch. The cost and the benefit of this memoization is that all calculated
quantities are effectively referentially transparent.
