\chapter{Computing}
\section{\susyv2 Analysis Framework}
\susyv2 is a standalone, \root-based analysis framework. Analysis
scripts are written in the Python programming language. These make use of
low-level, high-performance classes written in C++. This provides a good
compromise between speed and flexibility.

The \susyv2 package has been used for a number of analyses at
\ac{CMS}. These have primarily been \ac{SUSY}-based analyses. It has also been
used for the \PW polarisation measurement as well as several other projects. The
vast majority of the initial code was written by Dr. John Jones with subsequent
contributions from a number of others.

The \susyv2 code aims to minimise the number of reads performed on a \root tree
by reading branches on demand and performing lazy calculations as required to
satisfy analysis code requests for higher-level observables. The dependency
chain between calculated quantities may be viewed as a tree. The leaves of this
tree correspond to quantities stored directly in the \root file (or an
alternative serialisation format). To minimise computation, each node in this
tree performs its calculation (or \ac{IO} in the case of the leaf nodes) only
once per event. The results are then cached. Subsequent use of this quantity
then returns the cached result directly. Furthermore, access to quantities
dependent on others which have already been calculated will require the minimum
necessary calculation, reutilising cached values and minimising further \ac{IO}
or CPU usage.

As well as the performance advantages of this approach, it has the benefit of
enforcing a kind of ``referential transparency'' -- repeated access to a given
quantity must always yield the same result (at the single-event level). Whilst
this makes certain tasks more involved -- e.g. iterative cleaning of events --
it ensures that analysis selections must ``commute'' - since they are unable to
mutate any of the quantities on which they select. This emulates some of the
benefits available in purely functional programming languages such as Haskell.

My contributions were in the maintenance and development of this code-base, the
addition of a flexible Python-based configuration system, support for the \root
\textsc{TChain} class, infrastructure for managing and monitoring batch
submissions, and the implementation of a fast ``cross-cleaner''. The
cross-cleaner must resolve ambiguties between physics objects. The detected
ambiguities may form cyclic graphs, which require careful resolution.
