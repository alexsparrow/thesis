\chapter{Kinematics}
\label{app:kinematics}
A Lorentz tranformation can be written as
\begin{equation}
\left(\begin{array}{c} E' \\ p_{\parallel}' \end{array} \right)
=
\left(
\begin{array}{cc}
\gamma & -\gamma\beta \\
-\gamma\beta & \gamma
\end{array}
\right)
\left (\begin{array}{c} E \\ p_{\parallel} \end{array}\right),
\end{equation}
and
\begin{equation}
p_{\perp}' = p_{\perp},
\end{equation}
where $\gamma$ is the loretnz factor and $\beta=v/c$~\cite{pdg}.

Boosting from a particle's rest frame into the lab frame,
\begin{equation}
\left(\begin{array}{c} E \\ P \end{array} \right)
=
\left(
\begin{array}{cc}
\gamma & -\gamma\beta \\
-\gamma\beta & \gamma
\end{array}
\right)
\left (\begin{array}{c} M \\ 0 \end{array}\right),
\end{equation}
and so
\begin{eqnarray*}
E &=& \gamma M  \Longrightarrow \gamma = \frac{E}{M} \\
|P| &=& \gamma\beta M \Longrightarrow \beta = \frac{|P|}{\gamma M} = \frac{|P|}{E}.
\end{eqnarray*}
Also,
\begin{eqnarray*}
\gamma &=& \frac{\sqrt{P + M}}{M} \\
&=& \sqrt{1 +\left(\frac{|P|}{M}\right)^2}.
\end{eqnarray*}
