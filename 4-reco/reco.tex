\chapter{Physics Objects}
\section{Introduction}
In the previous chapter, details of the \ac{CMS} detector were presented. We
shall now begin to discuss the reconstruction algorithms used to derive analysis
level objects and quantities which will be of fundamental importance in later
chapters. The objects of primary interest for these purposes are the leptons,
jets and missing energy. The offline reconstruction algorithms used to
reconstruct each object will be presented, along with issues and properties
related to data analysis. Some details of the reconstruction performance at
\ac{CMS} will also be shown. Finally, the \acl{PF} method, which provides a
global reconstruction of the event, will be explained in some detail. As will be
seen, \ac{PF} combines tracking and calorimeter measurements to provide
excellent reconstruction of jets and missing energy.

\section{Leptons}
\subsection{Muons}
The full details of muon reconstruction at CMS are presented in
\cite{cms_mu_reco}. A brief overview will be presented here, focussing on the
aspects pertinent to the following analysis chapters. Muons are reconstructed in
both the muon chambers and the silicon tracker. To match this redundancy in the
measurement, a number of reconstruction algorithms are available.

\subsubsection{Standalone Muons}
Standalone muons are based solely on measurements in the muon chambers. The hits
in each chamber are fit individually to obtain seeds - a position and direction
vector along with an estimate of the transverse momentum. These form the basis
of a track fit in the muon chamber based on the Kalman-filter technique.
\subsection{Muons}
\subsection{Electrons}
\subsection{\ac{ECAL} Transparency Corrections}
\label{sec:reco_ecal_transparency}
\subsection{Muons}
\section{Jets}
\section{Missing Energy}

\section{Particle Flow at \ac{CMS}}
The \ac{PF} algorithm \cite{cms_pf_pas} attempts to provide a global
reconstruction of the event - accurate determination of the type, energy and
direction of all stable particles - by combining measurements from all
subdetectors in \ac{CMS}. This strategy is well suited for use with the \ac{CMS}
detector. The silicon tracker is able to reconstruct charged particle tracks
with high efficiency and low fake rate down to transverse momenta as low as
\unit{150}{\MeV}. Additionally, the granularity of the \ac{ECAL} is sufficient
for the separation of photons and charged particle energy deposits in jets with
\Pt of a few hundred \GeV. In contrast, the \ac{HCAL} is much coarser. However,
the combined energy resolution of both calorimeters is $\sim 10\%$ at
\unit{100}{\GeV}. This allows identification of the energy deposits associated
with neutral hadrons as an excess on top of that accounted for by matching the
deposits with charged tracks. The \ac{PF} algorithm is able to reconstruct the
components of jets and hadronic tau decays - primarily charged hadrons, neutral
hadrons and photons. This provides an improved measurement of the jet energy and
thus also of \MET.

The particle flow algorithm procedes by linking the tracks and energy clusters
to form blocks. A single block main contain some combination of a charged
particle track, one or more energy clusters and a muon. The fine granularity of
the \ac{CMS} detector ensures that blocks typically contain 1, 2 or 3 elements.
The links between each block are parameterised by a distance which encodes the
quality of the link. In order for \ac{PF} reconstruction to work correctly, it
is important that the inputs to the algorithm match certain requirements

\subsection{Iterative Tracking}
The measurements of momentum and direction of charged hadrons provided by the
tracker are hugely superior to those that can be provided by the calorimeters.
It is important therefore that the tracks input to the \ac{PF} procedure be
reconstructed with near 100\% efficiency. The fake rate must also be low to
avoid excess energy counting.

To meet these requirements, tracks are reconstructed using an iterative
algorithm. This begins by reconstructing tracks with very tight selection
requirements. Hits which can be unambiguously assigned in this step are then
removed from consideration and the remaining hits are reconstructed again, this
time with loosened selection criteria. This procedure is repeated with
progressively looser selection criteria ensuring high efficiency, whilst the
removal of hits at each stage reduces the fake rate induced by combinatorics.
After three iterations, tracks originating close to the beam line are
reconstructed with an efficiency of 99.5\% for muons and $>90\%$ for charged
hadrons. The fourth and fifth iterations relax constrains on the vertex,
reconstructing secondary charged particles.

\subsection{Calorimeter Clustering}
The success of the \ac{PF} reconstruction is dependent on certain aspects of the
clustering algorithm. In particular, as for the tracks, the clustering needs to
be highly efficient and be able to distinguish closely spaced energy deposits.
To this end, a specialised clustering algorithm was developed. This algorithm is
used the \ac{ECAL}, \ac{HCAL}, \ac{PS} but not in the \ac{HF} where each cell is
taken to be a cluster.

The first step of the algorithm produces seed clusters from local maxima in the
calorimeter cells above a given energy. The second step produces topological
clusters by extending cluster two include cells with at least one side in common
with the cluster and an energy above a threshold related to the standard
deviation of the electronics noise in the calorimeter. The topological clusters
are then transformed to particle flow clusters, with a separate particle flow
cluster for each seed comprising the topological cluster. The energy and
position of each particle flow cluster is determined iteratively with the energy
of each seed shared between the particle flow clusters.

\subsection{Building Links}
Each track is extrapolated from the position of its last measured hit to the
\ac{PS}, the \ac{ECAL} at a depth corresponding to the expected maximum for an
electron shower, the \ac{HCAL} at a depth of 1 interaction length. If the
extrapolated track position lies within the envelope of a cluster, a link is
created with a link distance equal to the $(\eta, \phi)$ distance between the
extrapolated track and the cluster. The envelope may be enlarged with respect to
the cluster by the extent of a single cell.

Additionally, energy contributions from bremsstrahlung photons are included by
extrapolating the track tangent at each tracker layer to the \ac{ECAL}. If the
extrapolated track lies within the envelope of the cluster, a link is created.
Links are also created between calorimeter clusters in different subdetectors if
the cluster position in the finer-grained calorimeter lies within the envelope
of the more coarsely grained calorimeter. The link distance is taken to be the
$(\eta, \phi)$ separation of the two clusters.

Muons are included when a global fit between a track in the tracker and a muon
track in the muon chambers yields an acceptable \chisq. If several global muons
are found for a single muon track, only that possessing the smallest \chisq is
retained - with the link distance determined by the \chisq.

\subsection{Particle Reconstruction}
The first step is to reconstruct muons. Each global muon gives rise to a
particle flow muon providing its momentum as determined from the global fit is
compatible with the track momentum to within 3 standard deviations. The
corresponding track is then removed from the block.

The next step is electron reconstruction. Electron tracks in the block are first
selected by a pre-identification step - electrons often leave short tracks and
lose energy via bremsstrahlung. Pre-identified electrons are then refit with a
Gaussian Sum Filter (see Section~TODO) and projected into the \ac{ECAL}.
Candidates passing tracking and calometric criteria are reconstructed as
particle flow electrons. The track and associated \ac{ECAL} clusters are then
removed from the block.

Tracks remaining are then subject to a tighter set of quality requirements, in
particular that the track \Pt uncertainty be smaller than the relative
calorimeter energy resolution for a charged hadron. Whilst some real hadrons are
lost by this requirement, the energy will be retained in the more accurate
measurement of the calorimeter.

Reconstruction of photons and neutral hadrons involves comparison of the track
momentum to the calorimetric energy. The cluster energies in the \ac{ECAL} are
callibrated for photons and the \ac{HCAL} for \unit{50}{\GeV} pions. For the
comparison to be valid, these must be recallibrated to account for
non-linearities in the \ac{HCAL} as well as the differing response of the
\ac{ECAL} to hadrons.

In the case that several tracks are linked to a single \ac{HCAL} cluster, the
total momenta of the tracks is compared to the callibrated calorimetric energy.
Tracks linked to multiple clusters are resolved by preserving the closest link
or links in certain cases. The track momentum is then compared to the total
callibrated calometric energy.

In the rare case that the energy is smaller than the track momentum by more than
three standard deviations, a relaxed search for fake tracks and global muons is
initiated. Global muons are identified as \ac{PF} muons if their momentum is
measured with an uncertainty below 25\%. Tracks are then progrssively removed
from the block, those with largest momentum uncertainty first, until either all
tracks with an uncertainty $>\unit{1}{\GeV}$ have been considered or the total
track momentum has decreased below the calorimetric energy. The reamining tracks
are interpreted as charged hadrons with momentum and energy taken from the track
momentum assuming the charged pion mass hypothesis. If the calorimeter energy
and track momentum are compatible within their uncertainties, the momentum is
redefined by a fit to both track momenta and the energy cluster. This is helpful
at very high energies, where the track parameters may be less well measured.

In the case that the callibrated energy is greater than the total track momentum
by more than the caloririmetric energy resolution, the excess is interpreted as
a photon and possibly a neutral hadron. If the excess is greater than the
\ac{ECAL} energy, a photon is created with this energy and the rest of the
excess interpreted as a neutral hadron. Otherwise, a photon only is
reconstructed from the uncallibrated \ac{ECAL} energy. This stems from the
observation that photons carry 25\% of the energy of a jet, with neutral hadrons
only 3\%.

Remaining \ac{ECAL} and \ac{HCAL} clusters not linked to a track (or for which
the associated track was disabled in the previous steps) are reconstructed as
photons and neutral hadrons respectively.

So called \ac{PF} jets are then reconstructed using a traditional jet clustering
algorithm (see Section~TODO) using the full set of particle flow objects.

\subsection{Physics Performance}
Two aspects of the performance of the \ac{PF} reconstruction are of relevance to
the analysis description that will follow: \acl{MET} measurement and jet
reconstruction.

\begin{figure}
\includegraphics[width=0.8\textwidth]{fig/pf_jet_mismatch}
\caption{Jet mismatch rate \cite{berger_left_handed_w}}
\label{fig:pf_jet_mismatch}
\end{figure}
