\chapter{Searching for Supersymmetry in the Single Lepton Channel at CMS}
\section{Introduction}
The \PW polarisation measurement, as well as being an interesting analysis in
its own right, also finds application to searches for \acl{NP}. The first of
these, comes from a more complete understanding of the \MET distribution in
\Wjets events - an important background to many \ac{SUSY} searches. The second,
is that the polarisation of the \PW coupled with the methods described in the
previous chapter provides a means to discrimate \ac{SUSY} events from \ac{SM}
backgrounds. This will be further explored in the following section.

\section{Comparing the Kinematics of \ac{SUSY} and \Wjets Events}
In the context of \Rparity conserving \ac{SUSY}, a typical event with a charged
lepton in the final state is expected to contain 3 invisible particles: two
\acp{LSP} and a neutrino. As a result, the total \MET in an event will often be
larger than the transverse momentum of the charged lepton and relatively
uncorrelated with it in terms of direction. In contrast, the large boost and
polarisation of a typical \PW decay lead to a more even balance between the \MET
and the transverse momentum of the charged lepton, as well as greater
correlation of their directions. These two consideration can be applied to both
\Wjets and \ttbar events - the two dominant backgrounds to a single lepton
\ac{SUSY} search.





\section{Monte Carlo Expectation}
\section{Systematics}
\section{Results}
