\chapter{Searching for Supersymmetry in the Single Lepton Channel at CMS}
\section{Introduction}
The \PW polarisation measurement, as well as being an interesting analysis in
its own right, also finds application to searches for \acl{NP}. The first of
these, comes from a more complete understanding of the \MET distribution in
\Wjets events - an important background to many \ac{SUSY} searches. The second,
is that the polarisation of the \PW coupled with the methods described in the
previous chapter provides a means to discrimate \ac{SUSY} events from \ac{SM}
backgrounds. This will be further explored in the following section.

\section{Separating \as{SUSY} and \ac{SM} Events}
\label{sec:susy_sm}
In the context of \Rparity conserving \ac{SUSY}, a typical event with a charged
lepton in the final state is expected to contain 3 invisible particles: two
\acp{LSP} and a neutrino. As a result, the total \MET in an event will often be
larger than the transverse momentum of the charged lepton and relatively
uncorrelated with it in terms of direction. In contrast, the large boost and
polarisation of a typical \PW decay lead to a more even balance between the \MET
and the transverse momentum of the charged lepton, as well as greater
correlation of their directions. These two consideration can be applied to both
\Wjets and \ttbar events - the two dominant backgrounds to a single lepton
\ac{SUSY} search.

In order to make use of the \PW polarisation effects described, this analysis
makes use of the \LP variable as described in Section~\ref{sec:wpol_lp}. For
events containing a charged lepton and \MET originating from a \PW decay with
large transverse momentum, the alignment of the charged lepton and neutrino
gives an \LP distribution confined to the range $[0,1]$. In contrast, for
\ac{SUSY} events the \MET component is larger than the lepton momentum and thus
the \PtW is likely to point in the direction of the \MET. Since the direction
of the charged lepton momentum and \MET will be mostly uncorrelated, \LP is
likely to tend to small values. Rewriting,
\begin{equation}
\LP = \frac{\Ptl}{\PtW}\cos\left(\PtW, \Ptl\right)
\end{equation}
it can be seen that in cases where the angle between the \MET and \Ptl is more
than 90\degrees, \LP will become negative.

 A second important difference between \ac{SUSY} and
\ac{SM} events is related to the overall energy scale of the events. As
discussed in Chapter~\ref{sec:susy}, \ac{SUSY} decays are expected to begin with
initial states much heavier than in \ac{SM} events. To provide some measure of
this energy scale without biasing the polarisation, the variable \STlep is
constructed as follows,
\begin{equation}
\STlep = |\Ptl| + |\MET|
\end{equation}
where it should be noted that \STlep is a scalar quantity. For \PW decays,
$\STlep \approx \PtW$. Since the energy scale of \ac{SUSY} is unknown, \STlep is
used to define search regions. This allows the search to be optimised without
introducing a strong dependence on the energy scale. For the purposes of this
analysis, 4 \STlep bins are employed. These are: $150 < \STlep < 250$, $250 <
\STlep < 350$, $350 < \STlep < 450$ and $\STlep > \unit{450}{\GeV}$. The lowest
region, $\STlep_{150}$ is taken to be at too low an energy scale to contain
\ac{SUSY} processes not excluded by previous searches. It is used as a sideband
to validate the analysis method.

\section{Analysis Method}
The \LP distributions for \ac{SM} backgrounds and two benchmark \ac{SUSY} models
are shown in Figure~\ref{fig:susy_lp}. Firstly it can be seen that the heuristic
discussion of the \LP shape given in Section~\ref{sec:susy_sm} is confirmed by
the full simulation with \ac{SM} events giving $\LP > 0$ and \ac{SUSY} events
clustering around $\LP \sim 0$.
\begin{figure}
\centering
\subfloat[]{\label{fig:susy_lp_el_st250}\includegraphics[width=0.3\textwidth]{fig/LP250_MCandSignal_El.pdf}}\quad
\subfloat[]{\label{fig:susy_lp_el_st350}\includegraphics[width=0.3\textwidth]{fig/LP350_MCandSignal_El.pdf}}\quad
\subfloat[]{\label{fig:susy_lp_el_st450}\includegraphics[width=0.3\textwidth]{fig/LP350_MCandSignal_El.pdf}}\\
\subfloat[]{\label{fig:susy_lp_mu_st250}\includegraphics[width=0.3\textwidth]{fig/LP250_MCandSignal_Mu.pdf}}\quad
\subfloat[]{\label{fig:susy_lp_mu_st350}\includegraphics[width=0.3\textwidth]{fig/LP350_MCandSignal_Mu.pdf}}\quad
\subfloat[]{\label{fig:susy_lp_mu_st450}\includegraphics[width=0.3\textwidth]{fig/LP350_MCandSignal_Mu.pdf}}
\caption[]{}
\label{fig:susy_lp}
\end{figure}

In order to make use of the discrimination power afforded by the \LP shape, a
signal and control region are defined. The signal region is defined such that an
enriched sample of \ac{SUSY} events is obtained without being highly model
dependent. It should be stressed that the intent is not to eliminate the
background altogether in this region. The control region likewise must select a
sample of \ac{SM} background events with sufficient statistics whilst guarding
against excessive signal contamination from \ac{SUSY} models. By studying the
\LP distribution across the \ac{CMSSM} parameter space, \LPsignal for the signal
region and \LPcontrol for the control region were found to be suitable choices.

To predict the \ac{SM} background contamination in the \LPsignal region, a
translation factor, \RCS is calculated in simulation. This is defined as
\begin{equation}
\RCS = \frac{N^{\textrm{MC}}(\LPsignal)}{N^{\textrm{MC}}(\LPcontrol)}
\end{equation}
where $N^{\textrm{MC}}(\LPsignal)$ and $N^{\textrm{MC}}(\LPcontrol)$ represent
the population, as calculated in simulation of \ac{SM} events in the signal and
control regions respectively. Once calculated, \RCS may be used along with a
measurement of the control region in data to predict the \ac{SM} background
contribution present in the signal region,
\begin{equation}
N^{\textrm{data}}(\LPsignal) = \RCS N^{\textrm{data}}(\LPcontrol)
\end{equation}
Whilst in principle it is possible to perform a more, or even completely,
data-driven prediction by performing a template fit to the \LP shape in the
control region and extrapolating into the signal region, this strategy was
explored for some time and found to be frustrated by inadequate statistics, even
at relatively large integrated luminosities.

The advantage of the translation factor \RCS is that by taking the ratio from
simulation, there is significant cancellation of many systematic uncertainties,
including in particular the jet energy scale that proved to be significant for
the \PW polarisation analysis (see Section~\ref{sec:wpol_syst_jec}). Since these
uncertainties do not cancel completely, they will be full evaluated in
Section~\ref{sec:susy_systematics} and are included in the statistical treatment
described in Chapter~\ref{sec:interpretation}.

One last point concerning \RCS is that, so that the prediction is not dominated
by uncertainties stemming from limited statistics in the control region, a value
of $\RCS << 1$ is preferable. As we will see, relatively small values of \RCS
are obtained using the definitions given above.

\section{Object Definitions}
The basic object selection requirements were defined to be consistent between
several complementary leptonic \ac{SUSY} searches at \ac{CMS}. They are
described and motivated further in \cite{susy_selection_an}.

\subsection{Jets and Missing Energy}
Jets and missing energy quantities are taken from the particle flow algorithm,
as described in Section~\ref{sec:reco_pf}. In addition, jets are required to
pass the ``LOOSE'' selection criteria, namely
\begin{itemize}
\item At least two particles - at least one of them charged - in the jet.
\item Fraction of jet energy carried by neutral hadrons less than 99\%.
\item Charged and neutral electromagnetic fractions both less than 99\%.
\end{itemize}
All jets are required to have a transverse momentum, $\Pt > \unit{40}{\GeV}$ and
are required to be within the fiducial region of the tracker, $|\eta| <
2.4$. The total hadronic transverse energy, \HT is calculated from jets passing
this selection.

\subsection{Muons}
Muon reconstruction is described in Section~\ref{sec:reco_muons}. Global muons
are selected with a number of additional quality requirements. These are similar
to thoseused in the \PW polarisation analysis (see Section~\ref{sec:wpol_muons})
with certain adjustements made to ensure consistency with other analyses or
adapt to the different analysis requirements.
\begin{itemize}
\item A normalized $\chi^2 < 10$ on the global muon fit
\item More than 10 hits in the tracker (including at least 1 pixel hit) and
  $\geq 2$ matching segments in the muon chambers.
\item A transverse distance to the nominal interaction point, $d_0 <
  \unit{200}{\micro\metre}$ and longitudinal distance to the primary vertex $d_z
  < \unit{1}{\centi\metre}$
\item The uncertainty on the muon transverse momentum $\sigma(\Pt)/\Pt^2 <
  \unit{0.001}{\reciprocal\GeV}$
\item Each global muon must also qualify as a tracker muon.
\item A combined relative isolation (see Eqn~\ref{eqn:wpol_mu_comb_iso}) $\CombIso < 0.1$.
\end{itemize}

\textbf{Tight} muons are defined by the requirements given above. \textbf{Loose}
muons use an identical selection but with the \CombIso cut loosened to 0.15.

\subsection{Electrons}
\textbf{Tight} electrons are reconstructed as described in
Section~\ref{sec:reco_electrons} using the 80\% efficiency working point but
with impact parameter requirements identical to those used for the
muons. \textbf{Loose} electrons use the 95\% efficiency working point with the
impact parameter criteria loosened as for the muon case.

\subsection{Resolving Ambiguities}
Since the leptons used in this analysis use the traditional reconstruction
methods at \ac{CMS} while jets and \MET are taken from the \ac{PF} algorithm,
ambiguities can exist. In order to avoid double counting, these ambiguities are
resolved by several cleaning steps.

To remove jets dominated by a lepton, any jet found within a cone of 0.1 (0.3)
of a selected muon (electron) is removed from consideration. In addition, muons
within a cone of 0.3 of any jet are rejected.

A second step corrects the \MET for differences between \ac{PF} and global muon
reconstruction. Each global muon is matched to a corresponding \ac{PF} muon
within a cone of $\DeltaR < 0.1$. The absolute relative difference between the
transverse momenta is then calculated. For cases where no match is found or this
difference is $> 20\%$, the event is rejected. For cases where the difference is
smaller than 20\%, the \MET receives a vectorial collection.

\section{Analysis Selection}
Selection begins with a set of event cleaning cuts common to many analyses at
\ac{CMS}. These address known detector and reconstruction problems as well as
supressing machine backgrounds. They are fully detailed in
\cite{susy_selection_an}.

Lepton selection requires exactly one \textbf{Tight} electron or muon. To remove
dilepton events and minimise overlap with searches in multilepton final states,
events containing a second \textbf{Loose} lepton are vetoed.

After the initial lepton selection cuts, events can enter two independent
samples. The first is a control sample, inverting the jet multiplicity cut to
obtain a sample known to be overwhelmingly dominated by \ac{SM} backgrounds. To
compensate, this sample is selected with a slightly relaxed \HT cut. The second
sample was used for the actual analysis. A jet multiplicity cut, $\Njets \geq 3$
is applied as well as an \HT cut.

The data-driven control sample was used to test the analysis techniques before
they were applied in the search dataset. During this time, the search dataset
was not studied (or ``blinded'') to avoid changes in the analysis procedure that
might bias the result. Once the analysis method was fully refined, the search
sample was ``unblinded'' and major changes to the analysis were no longer
allowed.

Due to the unavailability of suitable efficient and unbiased triggers, the
control sample was considered only for the muon channel. For the electron
channel, validation work was performed instead in the $150 < \STlep <
\unit{250}{\GeV}$ bin. Trigger thresholds in the control sample necessitated
increasing the transverse momentum cut on the muon to \unit{35}{\GeV}.

The full cutflow is shown in Table~\ref{tbl:susy_cutflow}.

\ctable[
cap=SUSY Search Cut Flow,
caption=SUSY Search Cut Flow,
mincapwidth=0.5\textwidth,
pos=h,
label=tbl:susy_cutflow
]{cc}{
}{\FL
Lepton Selection & Exactly one \textbf{Tight} electron or muon \NN
                 & $P_T^{\Pl} > \unit{20}{\GeV}$\NN
                 & $|\eta^{\Pgm}| < 2.1$, $|\eta^{\Pe}| < 2.5$\NN
Lepton Veto      & Zero additional leptons passing \textbf{Loose} criteria\ML
                 & $P_T^{\Pgm} > \unit{15}{\GeV}$, $P_T^{\Pe} > \unit{20}{\GeV}$\NN
                 & $|\eta^{\Pgm}| < 2.5$, $|\eta^{\Pe}| < 2.5$\NN
\multicolumn{2}{c}{Control Sample}\ML
Jets             & $< 3$~jets \NN
\HT              & $\HT > \unit{200}{\GeV}$ \ML
\multicolumn{2}{c}{Analysis Sample}\ML
Jets & $\geq 3$~jets \NN
\HT & $\HT > \unit{300}{\GeV}$\LL
}

\subsection{Monte Carlo Expectations}
\ctable[
cap=\ac{MC} event yields in the signal region,
caption={\ac{MC} event yields in the signal region, \LPsignal, normalised to
\unit{1.14}{\invfb}. Both muon and electron channels are shown. The contribution
from \ac{QCD} multi-jet events is expected to be negligible and thus is not shown.},
%pos=h!,
label=tbl:susy_mcexp_signal,
%doinside=\scriptsize
]{ccccccc}{
}{\FL
$\LP<0.15$          & \multicolumn{3}{c}{ Muons: \STlep range (GeV) } & \multicolumn{3}{c}{  Electrons: \STlep range (GeV) }\ML
Sample              & [250-350]                                         & [350-450]    & [450-$\inf$] & [250-350]   & [350-450]   & [450-$\inf$] \ML
\ttbar ($\ell$)     & 11.4$\pm$0.9                                      & 2.91$\pm$0.4 & 0.8$\pm$0.2  & 7.8$\pm$0.7 & 3.0$\pm$0.4 & 1.0$\pm$0.3\NN
\ttbar ($\ell\ell$) & 2.2$\pm$0.4                                       & 0.6$\pm$0.2  & 0.1$\pm$0.1
                    & 2.4$\pm$0.4                                       & 0.7$\pm$0.2  & 0.4$\pm$0.2\NN
W                   & 14.5$\pm$0.6                                      & 8.0$\pm$0.5  & 5.6$\pm$0.4
                    & 10.5$\pm$0.5                                      & 5.2$\pm$0.4  & 4.7$\pm$0.3\NN
Z                   & 0$\pm$1.5                                         & 0$\pm$1.5    & 0$\pm$1.5
                    & 0$\pm$1.5                                         & 0$\pm$1.5    & 0$\pm$1.5\NN
Total MC            & 28.1$\pm$1.1                                      & 11.5$\pm$0.7 & 6.5$\pm$0.4
                    & 20.8$\pm$1.0                                      & 8.8$\pm$0.6  & 6.1$\pm$0.5\NN
LM1                 & 24.2$\pm$0.9                                      & 23.1$\pm$0.9 & 16.2$\pm$0.7
                    & 22.9$\pm$0.9                                      & 20.8$\pm$0.8 & 14.7$\pm$0.7\NN
LM3                 & 24.8$\pm$0.8                                      & 16.7$\pm$0.6 & 9.7$\pm$0.5
                    & 22.8$\pm$0.7                                      & 14.8$\pm$0.6 & 9.7$\pm$0.5\NN
LM6                 & 1.9$\pm$0.0                                       & 2.5$\pm$0.1  & 5.9$\pm$0.1
                    & 1.7$\pm$0.0                                       & 2.3$\pm$0.1  & 5.3$\pm$0.1 \LL
}
\ctable[
cap=\acs{MC} event yields in the control region,
caption={\ac{MC} event yields in the control region, \LPcontrol, normalised to
\unit{1.14}{\invfb}. Both muon and electron channels are shown.},
%pos=h!,
label=tbl:susy_mcexp_control,
%doinside=\scriptsize
]{ccccccc}{
}{\FL
$\LP>0.30$          & \multicolumn{3}{c}{  Muons: \STlep range (GeV) } & \multicolumn{3}{c}{  Electrons: \STlep range (GeV) } \ML
Sample              & [250-350]                                        & [350-450]    & [450-$\inf$] & [250-350] & [350-450] & [450-$\inf$] \ML
\ttbar ($\ell$)     & 43.4$\pm$1.7                                     & 12.3$\pm$0.9 & 2.7$\pm$0.4
                    & 42.2$\pm$1.7                                     & 11.4$\pm$0.8 & 2.9$\pm$0.4\NN
\ttbar ($\ell\ell$) & 5.2$\pm$0.6                                      & 1.6$\pm$0.3  & 0.4$\pm$0.2
                    & 2.5$\pm$0.4                                      & 1.4$\pm$0.3  & 0.3$\pm$0.1\NN
W                   & 67.1$\pm$1.3                                     & 27.5$\pm$0.8 & 15.3$\pm$0.6
                    & 57.5$\pm$1.2                                     & 24.3$\pm$0.8 & 14.7$\pm$0.6\NN
Z                   & 0$\pm$1.5                                        & 1.7$\pm$1.5  & 0$\pm$1.5
                    & 7.5$\pm$3.6                                      & 0$\pm$0      & 0$\pm$0\NN
QCD                 & 0$\pm$1.5                                        & 0$\pm$1.5    & 0$\pm$1.5
                    & 10.4$\pm$3.0                                     & 7.2$\pm$1.7  & 3.8$\pm$0.7\NN
Total MC            & 116$\pm$2                                        & 43.4$\pm$2.3 & 18.4$\pm$0.8
                    & 120$\pm$5                                        & 44.3$\pm$2.1 & 21.7$\pm$1.1\NN
LM1                 & 2.8$\pm$0.3                                      & 1.4$\pm$0.2  & 0.8$\pm$0.2
                    & 2.9$\pm$0.3                                      & 2.0$\pm$0.3  & 1.3$\pm$0.2\NN
LM3                 & 9.7$\pm$0.5                                      & 4.2$\pm$0.3  & 2.3$\pm$0.2
                    & 9.1$\pm$0.5                                      & 4.2$\pm$0.3  & 2.5$\pm$0.2\NN
LM6                 & 0.5$\pm$0.0                                      & 0.4$\pm$0.0  & 0.9$\pm$0.0
                    & 0.5$\pm$0.0                                      & 0.4 $\pm$0.0 & 0.9$\pm$0.0\LL
}

\section{Triggers and Datasets}
Due to the construction of \STlep, events may be selected with moderate \MET
(and a high \Pt lepton) or large \MET (and a modertate \Pt lepton). This
necessitates a different trigger strategy to that used in other leptonic
\ac{SUSY} searches at \ac{CMS} which typically only select high \MET events.

For the search sample, a set of single-lepton cross-triggers are used, selecting
events with a single lepton in association with a large amount of hadronic
activity, \HT. As the luminosity increased during the 2011 run, it was necessary
to introduce a third requirement: a moderate cut on the \MET. The full list of
triggers used for both lepton channels are shown in
Table~\ref{tbl:susy_triggers}

\ctable[
cap=SUSY Triggers,
caption=SUSY Triggers,
pos=h!,
label=tbl:susy_triggers,
doinside=\scriptsize
]{ll}{
}{\FL
\multicolumn{2}{c}{\textbf{Search Sample}}\ML
\Pgm & HLT\_Mu8\_HT200\_v* \NN
     & HLT\_Mu15\_HT200\_v* \NN
     & HLT\_Mu15\_HT250\_PFMHT20\_v* \ML
\Pe  & HLT\_Ele10\_CaloIdL\_CaloIsoVL\_TrkIdVL\_TrkIsoVL\_HT200\_v* \NN
     & HLT\_Ele15\_CaloIdT\_CaloIsoVL\_TrkIdT\_TrkIsoVL\_HT250\_v* \NN
     & HLT\_HT250\_Ele5\_CaloIdVL\_TrkIdVL\_CaloIsoVL\_TrkIsoVL\_PFMHT35\_v* \NN
     & HLT\_HT300\_Ele5\_CaloIdVL\_TrkIdVL\_CaloIsoVL\_TrkIsoVL\_PFMHT40\_v* \ML
\multicolumn{2}{c}{\textbf{Control sample}}\ML
\Pgm & HLT\_Mu20\_v*, HLT\_IsoMu17\_v*\NN
     & HLT\_Mu30\_v*, HLT\_IsoMu24\_v*\LL
}

All signal and background Monte Carlo samples are from the Summer11 \ac{CMS}
\unit{7}{\TeV} production using \ac{CMSSW} 42. All processes are simulated using
the Madgraph matrix element generator, with the exception of the \ac{QCD} and
\ac{SUSY} signal samples which use \ac{PYTHIA} 6. All datasets, with the
exception of the \ac{SUSY} signal scan used to derive the limit use the full
detector simulation. The \ac{SUSY} signal scan uses the \ac{FASTSIM} simplified
simulation package to reduce processing time. All samples contain data-like
pile-up conditions, with a reweighting procedure used througout to reflect the
exact vertex multiplicity distribution in the data.


\section{Control Region}
In order to test that the simulation of electroweak background processes can be
relied upon for the calculation of the translation factor \RCS, the procedure is
first performed in the control sample. With jet multiplicity cut inverted, it is
not expected for new physics to appear significantly in this sample. It is
expected therefore that the background prediction in the signal region should
agree well with the observed signal yield. Furthermore, the level of agreement
between data and simulation is also important in establishing the method.

A summary of the yields in the \LPcontrol and \LPsignal regions in the control
sample is given in Table~\ref{tbl:susy_control_yields}. Shown are the yields per
subprocess from simulation - used to calculate the factor \RCS, the yields in
data and the resulting background prediction. Comparing the background
prediction, it is seen to agree within errors with the observe number of events
in the signal region. The uncertainties stem from the limited data statistics of
the control region and the limited Monte Carlo statistics used in the
calculation of \RCS.

\ctable[
caption=Event yields in the \LPsignal and \LPcontrol regions in the $<3$~jet
data control sample in the muon channel.,
pos=h!,
label=tbl:susy_control_yields,
doinside=\scriptsize
]{ccccccc}{
}{\FL
 & \multicolumn{3}{c}{Control Region: \LPcontrol} & \multicolumn{3}{c}{Signal Region: \LPsignal} \ML
Sample      & [250-350]    & [350-450]    & [450-$\inf$] & [250-350]             & [350-450]            & [450-$\inf$]\ML
\ttbar      & 50.1$\pm$1.8 & 7.8$\pm$0.7  & 2.8$\pm$0.4  & 10.5$\pm$0.8          & 2.8$\pm$0.4          & 0.7$\pm$0.2 \NN
W           & 959$\pm$24   & 162$\pm$9.7  & 46.2$\pm$5.2 & 83.7$\pm$7.0          & 22.8$\pm$3.7         & 12.3$\pm$2.8 \NN
Z           & 45.3$\pm$9.2 & 4.7$\pm$2.9  & 3.9$\pm$2.8  & 1.8$\pm$1.8           & 0$\pm$1.8            & 0$\pm$1.8 \NN
QCD         & 2.7$\pm$1.7  & 0.8$\pm$0.8  & 0$\pm$0.8    & 0$\pm$1.4             & 0$\pm$1.3            & 0$\pm$1.3 \NN
Total MC    & 1054$\pm$26  & 174$\pm$10.2 & 52.9$\pm$5.9 & 96$\pm$7.3            & 25.6$\pm$3.7         & 13$\pm$2.8 \ML
data        & 1051         & 179          & 52           & 92                    & 24                   & 11 \NN
SM Estimate &              &              &              & 95.8$\pm$10.2$\pm$7.6 & 26.3$\pm$5.5$\pm$4.1 & 12.8$\pm$4.0$\pm$3.0\ML
LM6         & 0.3$\pm$0.0  & 0.2$\pm$0.0  & 0.4$\pm$0.0  & 1.0$\pm$0.0           & 1.0$\pm$0.0          & 2.4$\pm$0.1 \LL
}


Comparisons of the variables \STlep, \MT and \Ptmu between data and simulation
are shown in Figure~\ref{fig:susy_mucontrol_kin}. A similar comparison is shown
for the \LP variable in bins of \STlep in
Figure~\ref{fig:susy_mucontrol_lp}. These distributions are those used to derive
the numbers shown in Table~\ref{tbl:susy_control_yields}. The data is seen to be
adequately described by the simulation.
\begin{figure}
\centering
\subfloat[]{\label{fig:susy_mucontrol_st}\includegraphics[width=0.3\textwidth]{fig/MuControl_ST150toInf}}\quad
\subfloat[]{\label{fig:susy_mucontrol_mt}\includegraphics[width=0.3\textwidth]{fig/MuControl_MT150toInf}}\quad
\subfloat[]{\label{fig:susy_mucontrol_pt}\includegraphics[width=0.3\textwidth]{fig/MuControl_MuPt150toInf}}
\caption[]{}
\label{fig:susy_mucontrol_kin}
\end{figure}

\begin{figure}
\centering
\subfloat[]{\label{fig:susy_mucontrol_lp250}\includegraphics[width=0.3\textwidth]{fig/MuControl_LP250}}\quad
\subfloat[]{\label{fig:susy_mucontrol_lp350}\includegraphics[width=0.3\textwidth]{fig/MuControl_LP350}}\quad
\subfloat[]{\label{fig:susy_mucontrol_lp450}\includegraphics[width=0.3\textwidth]{fig/MuControl_LP450}}
\caption[]{}
\label{fig:susy_mucontrol_lp}
\end{figure}

\section{Systematics}
\label{sec:susy_systematics}

\ctable[
  cap=Systematic uncertainties in the muon channel,
  caption=The relative effects on the values of \f0 and \fLmfR in the muon channel for the uncertainties described. The absolute values are shown in brackets.,
  label=tbl:wpol_mu_syst,
  doinside=\scriptsize
]{ c  p{2.5cm}  p{2.7cm}  p{2.5cm}  p{2.7cm}}{
}{\FL
     Uncertainty                         & $\fLmfR^{-}$                 & $\f0^{-}$                            & $\fLmfR^{+}$                & $\f0^{+}$ \ML
     \ac{JES}                            & $\pm11$\% (0.029)              & $\pm56$\% (0.123)       & $\pm3$\% (0.011)   & $\pm42$\% (0.092) \NN
     \MET Resolution                     & $\pm4$\% (0.012)               & $\pm3$\% (0.006)        & $\pm4$\% (0.012)   & $\pm2$\% (0.004)                        \NN
     Muon Scale $\pm$1\%/100 GeV   & $\mp$0.8\% (0.002)             & $\mp$ 11\% (0.004)      & $\pm1.2$\% (0.004) & $\mp$16.0\% (0.036)                   \ML
     Quadratic sum                       & $\pm12$\% (0.031)              & $\pm56$\% (0.123)       & $\pm5$\% (0.017)   & $\pm45$\% (0.099) \LL
}
\ctable[
  caption=The relative effects on the values of $f_{0}$ and $(f_{L} - f_{R})$ in the electron channel for the uncertainties described. The absolute values are shown in brackets.,
  label=tbl:elec_syst,
  doinside=\scriptsize
]{ l c c  c  c }{
}{\FL
                              & $(f_{L} - f_{R})^{-}$   & $f_{0}^{-}$           & $(f_{L} - f_{R})^{+}$      & $f_{0}^{+}$  \ML
PF Recoil Scale               &  $\pm$16\% (0.042) & $\pm$68\% (0.150)  & $\pm$9\% (0.027)  & $\pm$37\% (0.078)   \NN
PF Recoil Resolution          &  $\pm$18\% (0.046)   &  $\pm$21\% (0.047)   &  $\pm$12\% (0.037)    & $\pm$18\% (0.039)     \NN
Electron Scale Corrections $\pm$50\% &  $\pm$6.7\% (0.017) & $\pm$6.4\% (0.014)   & $\pm$6.1\% (0.019)     & $\pm$7.6\% (0.016)   \NN
BG Estimation                 &  $\pm$5.5\% (0.014)& $\pm$31.3\% (0.066)& $\pm$0.6\% (0.002) &   $\pm$1.4\% (0.003)   \NN
BG Estimation (Stat)          &  $\pm$2.8\% (0.007)& $\pm$17.1\% (0.036)& $\pm$0.6\% (0.002) &  $\pm$6.4\% (0.014)   \NN
Quadratic Sum                 &  $\pm$26\% (0.066)  & $\pm$79\% (0.174)   &  $\pm$16\% (0.050)  & $\pm$43\% (0.090) \LL
}

\section{Results}
\begin{figure}
\centering
\subfloat[]{\label{fig:susy_mu_lp250}\includegraphics[width=0.3\textwidth]{fig/MuHad_LP250}}\quad
\subfloat[]{\label{fig:susy_mu_lp350}\includegraphics[width=0.3\textwidth]{fig/MuHad_LP350}}\quad
\subfloat[]{\label{fig:susy_mu_lp450}\includegraphics[width=0.3\textwidth]{fig/MuHad_LP450}}\\
\subfloat[]{\label{fig:susy_el_lp250}\includegraphics[width=0.3\textwidth]{fig/ElHad_LP250}}\quad
\subfloat[]{\label{fig:susy_el_lp350}\includegraphics[width=0.3\textwidth]{fig/ElHad_LP350}}\quad
\subfloat[]{\label{fig:susy_el_lp450}\includegraphics[width=0.3\textwidth]{fig/ElHad_LP450}}
\caption[]{}
\label{fig:susy_datamc}
\end{figure}

\begin{figure}[htb]
\centering
\subfloat[]{\label{fig:susy_pred_mu}\includegraphics[width=0.4\textwidth]{fig/prediction_mu}}\quad
\subfloat[]{\label{fig:susy_pred_el}\includegraphics[width=0.4\textwidth]{fig/prediction_el}}\quad
\caption{Comparison of the number of events observed in the data and the
  expectations from the background estimation methods presented above, in the
  different \STlep bins.\subref{fig:susy_pred_mu} muon channel;
  \subref{fig:susy_pred_el} Electron channel.  The red error-bars indicate the
  statistical uncertainty of the data only.}
\label{fig:susy_pred}
\end{figure}