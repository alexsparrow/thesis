\chapter*{Introduction}
\addcontentsline{toc}{chapter}{Introduction}

Most philosophers of science would probably agree that the process by which
scientific revolutions unfold is understood either poorly or not at all. During
such periods, and by some poorly understood process, many of the foundational
ideas within a given field are rapidly replaced or altered. Often the field that
emerges, is so vastly different from what went before as to be almost
unrecognisable.

History is rich with examples across many scientific fields. The emergence of
quantum mechanics in the early part of the 20th century instigated a seismic
shift in physicists' understanding of the universe. This tore down many
long-held beliefs, but brought about massive progress -- both directly across
many scientific fields and indirectly with the development of modern
semiconductors.

In some ways, particle physics has become the victim of this great success. The
quantum revolution of the 1920's led eventually to the formulation of the
\acl{SM} of particle physics in the latter half of the 20th century. This theory
has been so phenomenally successful that it has lasted, almost unchanged, for
the last 30 years.

However, as is often the case in science, finding answers often yields further
questions. In the second decade of the 21st century, particle physics faces a
number of open questions which seem to be unanswerable within the context of the
\ac{SM}; most disturbingly, the failure to combine gravity and quantum field
theory in a single, unified theory. A related cause for concern has been the
mounting, and at this point fairly conclusive, evidence for dark matter -- a
substance lacking any explanation within currently accepted theories but
apparently constituting the greater part of the matter content of the universe.

It is for these reasons, that many feel another revolution in particle physics
is long overdue. Of course, theoretical developments have continued at a rapid
pace since the formulation of the \ac{SM}. This has bred a zoo of colourful
theories -- string theory and loop quantum gravity being among the current
favourites. However, without experimental results to guide theoretical progress,
such theories can remain only mathematical curiosities. The crucial missing
ingredient, many feel, is inspiration from the universe itself.

It is against this backdrop that the \ac{LHC}, a new generation particle
accelerator, has been constructed at the \ac{CERN} particle physics laboratory
near Geneva, Switzerland. Able to reach an unparalleled centre-of-mass energy of
\unit{14}{\TeV}, hopes are high that this will open new doors to understanding
the universe. Perhaps paradoxically, chief among its goals is the observation of
a \ac{SM} particle, the notorious Higgs boson, understood to give other
particles their mass. First conceived in the 1960's, the Higgs has proved
particularly elusive to experiment.

It may seem odd to hope for a radical reformation of the \ac{SM} whilst at the
same time seeking to complete its verification. However, this misses a crucial
point about the nature of scientific discoveries -- a null result is often the
most disruptive. A particularly well-known example in recent history is the
Michelson-Morley experiment. A null result utterly debunked the prevailing
theory at the time, the ``luminiferous ether'', allowing special relativity to
emerge as the superior theory of nature. Even if the Higgs is discovered at the
\ac{LHC}, as many expect it will be, measurement of its mass may provide
essential insight to the shape of physics beyond the \ac{SM}.

Many well-understood aspects of the \ac{SM} will be tested once again at the
\ac{LHC}. This is important for a number of reasons. Firstly, from a purely
pragmatic perspective, as a means of testing the various detectors before
venturing into unknown territory. Secondly, if history is any lesson, new
physics can often appear in completely unexpected places.

The first part of the work in this thesis presents the results of just such a
precision measurement -- the measurement of the transverse polarisation of the
\PW boson. Surprisingly this measurement is novel not just because of the
increased energy of the \ac{LHC}. Rather, for reasons that will be explained,
the proton-proton environment at the \ac{LHC} leads to a dominance of
left-handed over right-handed \PW bosons at large transverse momentum.

The tools developed for this measurement turn out to be more generally useful in
searches for new physics. The second part of this thesis, presents a search for
a popular extension to the \ac{SM} known as \acl{SUSY}. This is theoretically
attractive for a number of reasons, not least that it provides a potential
solution to the Dark Matter problem. Whilst \ac{SUSY} is a fairly incremental
step beyond the \ac{SM}, and not the drastic paradigm shift many feel is
necessary, it can be seen as a step on the way to such a discovery. Most
excitingly for the \ac{LHC}, it makes solid, testable predictions in the form of
a myriad of undiscovered particles. As will be seen, there are good reasons for
believing that if these particles exist, they should be produced at the
\ac{LHC}.

The structure of this thesis is as follows. \chap~\ref{sec:sm} begins with an
introduction to the \ac{SM} and an explanation of the Higgs mechanism. The
deficiencies of the \ac{SM} are highlighted and the solutions provided by
supersymmetry are presented in
\chap~\ref{sec:susy}. \chap~\ref{sec:framework} introduces the theoretical
background necessary for understanding the \PW polarisation measurement. Models
of supersymmetry useful for interpreting experimental results will also be
discussed.

\chap~\ref{sec:experiment} will introduce the \ac{LHC} and the \ac{CMS}
experiment -- a large, general purpose detector designed for new physics searches
as well as precise tests of the \ac{SM}. \chap~\ref{sec:reco} will then
discuss the object reconstruction and selection requirements used for the
analysis work presented in later chapters.

\chap~\ref{sec:wpol} then presents the \PW polarisation analysis with
particular attention paid to the electron channel of the measurement and the
fitting procedure which was the focus of the work in this
thesis. \chap~\ref{sec:susysearch} then describes how this analysis was
adapted to search for supersymmetry in events with a single lepton, jets and
missing energy. Finally, \chap~\ref{sec:interpretation} takes the results of
the \ac{SUSY} search and provides interpretation within the context of the
various models detailed in \chap~\ref{sec:framework}.
