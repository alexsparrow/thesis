\chapter*{Introduction}
\addcontentsline{toc}{chapter}{Introduction}
\markboth{Introduction}{Introduction}

Most philosophers of science would probably agree that the process by which
scientific revolutions unfold is understood either poorly or not at all. During
such periods, and by some mysterious process, many of the foundational ideas
within a given field are rapidly replaced or altered. Often the field that
emerges is so vastly different from what went before as to be almost
unrecognisable.

History is rich with examples across many scientific fields. A particularly
spectacular case was the emergence of quantum mechanics in the early part of the
20th century. This instigated a seismic shift in physicists' understanding of
the universe, tearing down many long-held beliefs, and bringing with it
incredible progress.

In some ways, particle physics has become the victim of this great success. The
quantum revolution of the 1920's led eventually to the formulation of the
\acf{SM} of particle physics in the latter half of the 20th century. This theory
has been so phenomenally successful that it has lasted, almost unchanged, for
the last 30 years.

However, as is often the case in science, finding answers often yields further
questions. In the second decade of the 21st century, particle physics faces a
number of open questions, which seem to be unanswerable within the context of
the \ac{SM}; most disturbingly, the failure to combine gravity and quantum field
theory in a single, unified theory. A related cause for concern has been the
mounting, and at this point fairly conclusive, evidence for dark matter -- a
substance lacking any explanation within the \ac{SM}.

It is for these reasons that many feel another revolution is long overdue. Of
course, theoretical developments have continued at a rapid pace since the
formulation of the \ac{SM}. This has bred a zoo of colourful theories -- string
theory and loop quantum gravity being among the current favourites. However,
without experimental results to guide theoretical progress, such theories can
remain only mathematical curiosities. The crucial missing ingredient, many feel,
is inspiration from the universe itself.

It is against this backdrop that the \ac{LHC}, a next generation particle
accelerator, has been constructed at the \ac{CERN} particle physics laboratory
near Geneva, Switzerland. Able to reach an unparalleled centre-of-mass energy of
\unit{14}{\TeV}, hopes are high that this will open new doors to understanding
the universe. Perhaps paradoxically, chief among its goals is the observation of
a \ac{SM} particle, the notorious Higgs boson, understood to give other
particles their mass. First conceived in the 1960's, the Higgs has proved
particularly elusive to experiment.

It may seem strange to hope for a radical reformation of the \ac{SM}, whilst at
the same time seeking to discover its last missing piece. However, this misses a
crucial point about the nature of scientific discoveries -- a null result is
often the most disruptive. A particularly well-known example in recent history
is the Michelson-Morley experiment~\cite{michelson_morley}. In this case, a null result utterly debunked
the prevailing theory at the time, the ``luminiferous ether''. This allowed
special relativity to emerge as a superior theory of nature. Even if the Higgs
is discovered at the \ac{LHC}, as many expect it will be, measurement of its
mass may provide essential insight to the shape of physics beyond the \ac{SM}.

Many well-understood aspects of the \ac{SM} will be tested once again at the
\ac{LHC}. This is important for a number of reasons. Firstly, from a purely
pragmatic perspective, as a means of testing the various detectors before
venturing into unknown territory. And secondly because, if history is any
lesson, new physics can often appear in completely unexpected places.

The first part of the work in this thesis presents the results of just such a
precision measurement -- the polarisation of the \PW boson. Surprisingly the
novelty of this measurement is not just due to the increased energy of the
\ac{LHC}. Rather, for reasons that will be explained, the proton-proton
environment at the \ac{LHC} leads to a dominance of left-handed over
right-handed \PW bosons at large transverse momentum. This is an effect that has
not been previously observed.

The tools developed for this measurement turn out to be more generally useful in
searches for new physics. The second part of this thesis presents a search for a
popular extension to the \ac{SM}, known as \acf{SUSY}. This is theoretically
attractive for a number of reasons, not least that it provides a potential
solution to the Dark Matter problem. Whilst \ac{SUSY} is a fairly incremental
step beyond the \ac{SM}, and not the drastic paradigm shift many feel is
necessary, it can be seen as a stepping-stone to further discovery. Most
excitingly for the \ac{LHC}, it makes solid, testable predictions in the form of
a myriad of undiscovered particles. If these particles exist, they should be
observed at the \ac{LHC}.

The structure of this thesis is as follows. \chap~\ref{sec:sm} begins with an
introduction to the \ac{SM} and an explanation of the Higgs
mechanism. \chap~\ref{sec:susy} will highlight the deficiencies of the \ac{SM}
and present \ac{SUSY} as a possible solution. \chap~\ref{sec:framework}
introduces the theoretical background necessary for understanding the \PW
polarisation measurement. Models of \ac{SUSY} useful for interpreting
experimental results will also be discussed.

\chap~\ref{sec:experiment} will introduce the \ac{LHC} and the \ac{CMS}
experiment -- a large, general purpose detector designed for new physics
searches as well as precision tests of the \ac{SM}. \chap~\ref{sec:reco} will
then discuss the object reconstruction and selection requirements used for the
analysis work presented in later chapters.

In \chap~\ref{sec:wpol} the \PW polarisation analysis will be
presented. Particular attention will be paid to the electron channel of the
measurement and the fitting procedure which was the focus of the author's
work. \chap~\ref{sec:susysearch} will then describe how this analysis was
adapted to search for \ac{SUSY} in events with a single lepton, jets and missing
energy. Finally, \chap~\ref{sec:interpretation} will take the results of this
search and provide interpretation within the context of several theoretical
models.
